\documentclass[main.tex]{subfiles}

\begin{document}

\section{Application to presheaves}
\label{sec:application_to_presheaves}

\subsection{Introduction}
\label{ssc:introduction}

The goal of the remainder of this work is to construct a lax monoidal functor
\begin{equation}
  \label{eq:rhat}
  \hat{r}\colon \Span(\S) \to \ICat,
\end{equation}
which takes a space $X$ to the category $\P(X)$ of presheaves on $X$, and a span of spaces
\begin{equation*}
  \begin{tikzcd}
    & Y
    \arrow[dl, swap, "g"]
    \arrow[dr, "f"]
    \\
    X
    && X'
  \end{tikzcd}
\end{equation*}
to the map $\P(X) \to \P(X')$ given by the composition
\begin{equation*}
  \begin{tikzcd}
    & \P(Y)
    \arrow[dr, "f^{*}"]
    \\
    \P(X)
    \arrow[ur, "g_{!}"]
    \arrow[rr, "f^{*} \circ g_{!}"]
    && \P(X').
  \end{tikzcd}.
\end{equation*}

In this section, we leverage \hyperref[thm:main]{Theorem~\ref*{thm:main}} to construct $\hat{r}$. First, we give a sketch of the process, ignoring the lax monoidal structure.

\subsubsection{Sketch of the construction of the functor underlying \texorpdfstring{$\hat{r}$}{r}}
\label{sss:sketch_of_underlying_functor}

Consider the category $\Map(\Delta^{1}, \S)$, whose objects are arrows $S \to X$ in $\S$, and whose morphisms $(S \to X) \to (S' \to X')$ are commuting squares in $\S$
\begin{equation*}
  \begin{tikzcd}
    S
    \arrow[r]
    \arrow[d]
    & S'
    \arrow[d]
    \\
    X
    \arrow[r]
    & X'
  \end{tikzcd}.
\end{equation*}

There is a functor $\ev_{1}\colon \Map(\Delta^{1}, \S) \to \S$, which sends an $n$-simplex $\sigma\colon \Delta^{n} \to \Map(\Delta^{1}, \S)$ corresponding to a map $\Delta^{n} \times \Delta^{1} \to \S$ to the pullback $n$-simplex 
\begin{equation*}
\begin{tikzcd}
  \Delta^{n} \times \Delta^{\{1\}} 
  \arrow[r, hook]
  & \Delta^{n} \times \Delta^{1}
  \arrow[r, "\sigma"]
  & \S
\end{tikzcd}.
\end{equation*}
In other words, this map sends a prism to its bottom face.

In \cite{highertopostheory} the following is shown.
\begin{proposition}
  The functor
  \begin{equation}
    \label{eq:ev1}
    \ev_{1}\colon \Map(\Delta^{1}, \S) \to \S
  \end{equation}
  is an inner fibration. Further, for any morphism $f$ between objects $(S \to X)$ and $(S' \to X')$ in $\Map(\Delta^{1}, \S)$ corresponding to a square
  \begin{equation*}
    \sigma =
    \begin{tikzcd}
      S
      \arrow[r]
      \arrow[d]
      & S'
      \arrow[d]
      \\
      X
      \arrow[r]
      & X'
    \end{tikzcd}
  \end{equation*}
  in $\S$, the following hold.
  \begin{itemize}
    \item The morphism $f$ is $\ev_{1}$-cartesian if and only if $\sigma$ is pullback.

    \item The morphism $f$ is $\ev_{1}$-cocartesian if and only if the morphism $S \to S'$ in $\sigma$ is an equivalence.
  \end{itemize}
\end{proposition}
\begin{proof}
  \cite[Lem.~6.1.1.1]{highertopostheory} and \cite[Lemma 2.4.7.12]{highertopostheory}.
\end{proof}

With this in mind, the following follows immediately.

\begin{corollary}
  The functor $\ev_{1}$ of \hyperref[eq:ev1]{Equation~\ref*{eq:ev1}} is a bicartesian fibration.
\end{corollary}
\begin{proof}
  Fix a generic morphism $f\colon X \to X'$ in $\S$.

  First, we check that this morphism has a cartesian lift given a lift of its target. Fix a lift $S' \to X'$ of $X'$ in $\Map(\Delta^{1}, \S)$. This gives us the following data.
  \begin{equation*}
    \begin{tikzcd}
      & S'
      \arrow[d]
      \\
      X
      \arrow[r]
      & X'
    \end{tikzcd}
  \end{equation*}

  We can always find a cartesian lift because the pullback
  \begin{equation*}
    \begin{tikzcd}
      S' \times_{X} X'
      \arrow[r]
      \arrow[d]
      & S'
      \arrow[d]
      \\
      X
      \arrow[r, "f"]
      & X'
    \end{tikzcd}
  \end{equation*}
  always exists.

  Now we check that we can find a cocartesian lift given the data
  \begin{equation*}
    \begin{tikzcd}
      S
      \arrow[d, "g"]
      \\
      X
      \arrow[r, "f"]
      & X'
    \end{tikzcd}.
  \end{equation*}
  Such a lift is given, for example, by the square
  \begin{equation*}
    \begin{tikzcd}
      S
      \arrow[d, swap, "g"]
      \arrow[r, equals]
      & S'
      \arrow[d, "f \circ g"]
      \\
      X
      \arrow[r, "f"]
      & X'
    \end{tikzcd}.\qedhere
  \end{equation*}
\end{proof}

Thus, for each edge $f\colon \Delta^{1} \to \S$, corresponding to a morphism $f\colon X \to Y$ of spaces, the fiber of $\ev_{1}$ over $f$ corresponds to a pair of adjoint functors
\begin{equation*}
  f_{!} : \S^{/X} \longleftrightarrow \S^{/Y} : f^{*}.
\end{equation*}
Interpreting $\S^{/X}$ as models for $\P(X)$, the $\infty$-categories of presheaves on $X$, the functor $f^{*}$ corresponds to the functor which restricts along $f$, taking a presheaf $Y \to \S$ to the pullback $X \to Y \to \S$. The functor $f_{!}$ takes a preasheaf $X \to \S$ to its left Kan extension along $X \to Y$.

This gives us everything we need to build our functor $\hat{r}$. By taking spans in $\Map(\Delta^{1}, \S)$ and $\S$ (and appropriately restricting the legs), \hyperref[thm:main]{Theorem~\ref*{thm:main}} will provide us with a cocartesian fibration over $\Span(\S)$, which will classify a functor $\Span(\S) \to \ICat$. We thus define the following triple structures.
\begin{itemize}
  \item In $\S$, we place no restriction on the frontward- or backward-facing legs of the spans, taking the triple $(\S, \S\downdag = \S, \S\updag = \S)$.

  \item In $\Map(\Delta^{1}, \S) = \mathcal{M}$, we allow arbitrary egressive morphisms, taking $\category{M}\updag = \category{M}$. For ingressive morphisms, we take only those morphisms $(S \to X) \to (S' \to X')$ corresponding to squares of the form
    \begin{equation*}
      \begin{tikzcd}
        S
        \arrow[r, "\simeq"]
        \arrow[d]
        & S'
        \arrow[d]
        \\
        X
        \arrow[r]
        & X'
      \end{tikzcd}
    \end{equation*}
    in $\category{C}$. That is, we take $\mathcal{M}\downdag$ to be the subcategory of $\mathcal{M}$ on morphisms of this form.
\end{itemize}

One checks the conditions of \hyperref[thm:main]{Theorem~\ref*{thm:main}} and finds that this yields a cocartesian fibration
\begin{equation*}
  \Span\triple{M} \to \Span(\S).
\end{equation*}
The functor $\Span(\S) \to \ICat$ classified by this cocartesian fibration has the following description.
\begin{itemize}
  \item Each object $X \in \Span(\S)$ is sent, up to equivalence, to the fiber over $X$. This is equivalent to $(\S^{/X})\op \simeq (\P(X))\op$.

  \item A morphism in $\Span(\S)$ corresponding to a span $X \leftarrow Y \rightarrow X'$ in $\S$ sends an object $S \to X$ to the object $S' \to X'$ defined as in the following diagram in which the left-hand square corresponds to an $\ev_{1}$-cartesian morphism, and the right-hand square corresponds to an $\ev_{1}$-cocartesian morphism.
    \begin{equation*}
      \begin{tikzcd}
        S
        \arrow[d]
        & S \times_{X} Y
        \arrow[l]
        \arrow[r, "\simeq"]
        \arrow[d]
        & S'
        \arrow[d]
        \\
        X
        & Y
        \arrow[l, "q"]
        \arrow[r, "p"]
        & X'
      \end{tikzcd}
    \end{equation*}
\end{itemize}

Denote by $\mathcal{F}$ the presheaf $X \to \S$ classified by $S \to X$. Then the presheaf classified by $S \times_{X} Y \to Y$ classifies the presheaf $q^{*}\mathcal{F}$, and the presheaf classified by $S' \to X'$ classifies the presheaf $p_{!}q^{*}\mathcal{F}$. This is precisely what we wanted.

There is a small annoyance: the functor we have constructed sends a space $X$ not to the category $\P(X)$ of presheaves on $X$, but to $\P(X)\op$. This is an artifact of the way we constructed our category of spans, and is not an essential difficulty; composint with $\mathrm{op}\colon \ICat \to \ICat$ gives us a functor sending $X \mapsto \P(X)$.

\subsubsection{Sketch of the construction of the lax monoidal structure}
\label{sss:sketch_of_lax_monoidal_structure}

Note that the category $\S$ has products. This induces a symmetric monoidal structure on $\S$, the cartesian monoidal structure (defined, for example, in \cite[Sec.~2.4.1]{luriehigheralgebra}), which (as we will show) induces a symmetric monoidal structure on $\Span(\S)$. When we take $\Cat_{\infty}$ also to carry the cartesian monoidal structure, we will see that our functor $\hat{r}$ (\hyperref[eq:rhat]{Equation~\ref*{eq:rhat}}) even admits a lax monoidal structure.

In the theory of quasicategories, one packages up the coherence data involved in writing down a symmetric monoidal category using a cartesian fibration, and a

The idea behind our construction of the functor $\hat{r}$, in broad strokes, is as follows.

In \hyperref[ssc:defining_the_maps_p_q_and_r]{Subsection~\ref*{ssc:defining_the_maps_p_q_and_r}}, we define symmetric monoidal structures on $\Map(\Delta^{1}, \S)$ and $\S$, and upgrade the functor $\ev_{1}$ to a symmetric monoidal functor $r$ as follows (the monoidal structures on $\Map(\Delta^{1}, \S)^{\otimes}$ and $\S^{\otimes}$ are not exactly the standard cocartesian monoidal functors, as we will later discuss; they are represented by \emph{cartesian,} rather than \emph{cocartesian} fibrations). Schematically, this means we will define the following data.
\begin{equation}
  \label{eq:wishlist_prq}
  \begin{tikzcd}
    \Map(\Delta^{1}, \S)^{\otimes}
    \arrow[rr, "r"]
    \arrow[dr, swap, "q"]
    && \S^{\otimes}
    \arrow[dl, "p"]
    \\
    & \Finp\op
  \end{tikzcd}
\end{equation}

\begin{note}
  Here, and in what follows, we notationally suppress the nerve in $N(\Finp)$, writing simply $\Finp$.
\end{note}

In \hyperref[ssc:restricting_the_legs_of_the_spans]{Subsection~\ref*{ssc:restricting_the_legs_of_the_spans}} and \hyperref[ssc:building_categories_of_spans]{Subsection~\ref*{ssc:building_categories_of_spans}}, we take spans in each of the categories in \hyperref[eq:wishlist_prq]{Diagram~\ref*{eq:wishlist_prq}}, restricting the backwards- and forwards-facing legs judiciously so that we get a diagram
\begin{equation}
  \label{eq:wishlist_pivarpirho}
  \begin{tikzcd}
    \Span(\Map(\Delta^{1}, \S))^{\otimes}
    \arrow[rr, "\rho"]
    \arrow[dr, swap, "\varpi"]
    && \Span(\S)^{\otimes}
    \arrow[dl, "\pi"]
    \\
    & \Finp
  \end{tikzcd},
\end{equation}
where the $\varpi$ and $\pi$ are symmetric monoidal structures and $\rho$ is a cocartesian fibration. In the language of \cite{luriehigheralgebra}, $\rho$ exhibits $\Span(\Map(\Delta^{1}, \S))$ as a $\Span(\S)$-monoidal category.

In \hyperref[ssc:building_the_functor]{Subsection~\ref*{ssc:building_the_functor}}, we will finally construct the functor $\hat{r}$. Restricting to the component of $\rho$ over $\langle 1 \rangle \in \Finp$, one finds precisely the cocartesian fibration described in \hyperref[sss:sketch_of_underlying_functor]{Subsubsection~\ref*{sss:sketch_of_underlying_functor}}. However, by \cite{luriehigheralgebra}, 2.4.2.4--2.4.2.6, the existence of $\rho$, $\pi$, and $\varpi$ implies that the functor classified by this cartesian fibration is lax monoidal.

Much of the technology in this section is similar to work already done in \cite{spectralmackeyfunctors2}. However, the manner in which we present it allows for generalization in a different direction than that carried out there. This will be the subject of future work.

\subsection{Defining the maps \texorpdfstring{$p$}{p}, \texorpdfstring{$q$}{q}, and \texorpdfstring{$r$}{r}}
\label{ssc:defining_the_maps_p_q_and_r}

In this section, we begin our construction of the functor $\hat{r}$, following the plan laid out in \hyperref[sss:sketch_of_lax_monoidal_structure]{Subsubsection~\ref*{sss:sketch_of_lax_monoidal_structure}}. Our first step is to define the maps $p$, $q$, and $r$ as described in \hyperref[eq:wishlist_prq]{Diagram~\ref*{eq:wishlist_prq}}. The maps $p$ and $q$ are cartesian fibrations, exhibiting symmetric monoidal structures on $\Map(\Delta^{1}, \S)$ and $\S$; the map $r$ is a monoidal version of the map $\ev_{1}$.

Recall that for any $\infty$-category $\category{C}$ with finite products, one can construct a symmetric monoidal structure $\category{C}^{\times} \to \Finp$ whose corresponding tensor product $\times\colon \category{C} \times \category{C} \to \category{C}$ is the cartesian product; for more details, see \cite[Sec.\ 2.4.1]{luriehigheralgebra}. Similarly, as described in \cite[Sec.\ 2.4.2]{luriehigheralgebra}, for any $\infty$-category with coproducts there is a symmetric monoidal structure $\category{C}^{\amalg} \to \Finp$, called the \emph{cocartesian monoidal structure,} whose tensor product $\amalg\colon \category{C} \times \category{C} \to \category{C}$ is the coproduct. Because cocartesian fibrations give better control of the morphisms out of an object than morphisms into an object, the cocartesian monoidal structure is much more tractable than the cartesian monoidal structure.

Since taking coproducts in $\category{C}\op$ is the same as taking products in $\category{C}$, we can express the cartesian monoidal structure on $\category{C}$ as a cartesian fibration rather than a cocartesian fibration by taking the opposite of the cocartesian monoidal structure on $\category{C}\op$. In the case $\category{C} = \S$, this gives us the monoidal structure we are interested in.

\begin{definition}
  We define the map $p\colon \S_{\times} \to \Finp\op$ to be the functor
  \begin{equation*}
    ((\S\op)^{\amalg} \to \Finp)\op,
  \end{equation*}
  where $(\S\op)^{\amalg} \to \Finp$ is the cocartesian monoidal structure, as described in \cite[Sec.\ 2.4.2]{luriehigheralgebra}.
\end{definition}

In low degrees, the category $\S_{\times}$ admits the following description.
\begin{itemize}
  \item The objects are pairs $(\langle n \rangle, \vec{X})$, where $\langle n \rangle$ is an object of $\Finp$, and $\vec{X} = [X_{1}, \ldots, X_{n}]$ is an ordered $n$-tuple of objects of $\S$.

  \item The morphisms $(\langle m \rangle, \vec{X}) \to (\langle n \rangle, \vec{Y})$ are pairs $(\phi, \Psi)$, where $\phi\colon \langle n \rangle \to \langle m \rangle$ is a morphism in $\Finp$, and $\Psi$ consists, for each $i \in \langle m \rangle^{\circ}$, of a collection of morphisms $X_{i} \to Y_{j}$ for each $j \in \langle n \rangle^{\circ}$ with $\phi(j) = i$.
\end{itemize}

The functor $\S_{\times} \to \Finp\op$ is the obvious forgetful functor.

\begin{example}
  \label{eg:morphisms_in_cocartesian_monoidal_structure_on_spaces}
  An object $(\langle 1 \rangle, \vec{X})$ of $\S_{\times}$ in the fiber over $\langle 1 \rangle \in \Finp\op$ is the same thing as an object $X \in \langle S \rangle$. An object $( \langle 2 \rangle, \vec{Y})$ in the fiber over $\langle 2 \rangle$ is the same as a pair of objects $[Y, Y']$ in $\S$.

  A morphism $(\phi, \Psi)\colon (\langle 1 \rangle, \vec{X}) \to (\langle 2 \rangle, \vec{Y})$ where $\phi\colon \langle 2 \rangle \to \langle 1 \rangle$ is the morphism in $\Finp$ which sends $1 \mapsto 1$, $2 \mapsto 1$, consists of a pair of maps $X \to Y$, $X \to Y'$.
  \begin{equation*}
    \begin{tikzcd}[row sep=tiny]
      & 1
      \arrow[dl, mapsto]
      \\
      1
      \\
      & 2
      \arrow[ul, mapsto]
    \end{tikzcd}
    \qquad\qquad
    \begin{tikzcd}[row sep=tiny]
      & Y
      \\
      X
      \arrow[ur]
      \arrow[dr]
      \\
      & Y'
    \end{tikzcd}
  \end{equation*}
\end{example}

\begin{lemma}
  The map $p$ is a cartesian fibration, and a morphism $(\phi, \Psi)\colon (\langle m \rangle, \vec{X}) \to (\langle n \rangle, \vec{Y})$ in $\S_{\times}$ is $p$-cartesian if and only if it for each $i \in \langle m \rangle^{\circ}$, the maps $X_{i} \to Y_{j}$ exhibit $X_{i}$ as the product of the $Y_{j}$.
\end{lemma}
\begin{proof}
  This follows easily from \cite[Rem.~2.4.3.4]{luriehigheralgebra}.
\end{proof}

\begin{example}
  The morphism described in \hyperref[eg:morphisms_in_cocartesian_monoidal_structure_on_spaces]{Example~\ref*{eg:morphisms_in_cocartesian_monoidal_structure_on_spaces}} is cartesian if and only if $X \simeq Y \times Y'$, and the maps $X \to Y$ and $X \to Y'$ are the canonical projections.
\end{example}

We can now define the maps $q$ and $r$ rather easily.

\begin{definition}
  We define maps $q$ and $r$ via the following diagram, where the left-hand square is pullback. In particular, the space $\Map(\Delta^{1}, \S)_{\times}$ is defined to be the (strict) pullback below.
  \begin{equation*}
    \begin{tikzcd}
      \Map(\Delta^{1}, \S)_{\times}
      \arrow[r, hook]
      \arrow[rr, bend left, "r"]
      \arrow[d, swap, "q"]
      & \Map(\Delta^{1}, \S_{\times})
      \arrow[r, "\ev_{1}"]
      \arrow[d]
      & \S_{\times}
      \arrow[d, "p"]
      \\
      \Finp\op
      \arrow[r, hook]
      \arrow[rr, bend right, swap, "\id"]
      & \Map(\Delta^{1}, \Finp\op)
      \arrow[r, "\ev_{1}"]
      & \Finp\op
    \end{tikzcd}
  \end{equation*}
\end{definition}

\begin{lemma}
  The map $q$ is a cartesian fibration, and the cartesian morphisms are those which are level-wise cartesian morphisms in $\S_{\times}$.
\end{lemma}
\begin{proof}
  The map $q$ is given by the pullback
  \begin{equation*}
    \begin{tikzcd}
      \Map(\Delta^{1}, \category{S})_{\times}
      \arrow[r, hook]
      \arrow[d, swap, "q"]
      & \Map(\Delta^{1}, \category{S}_{\times})
      \arrow[d, "q'"]
      \\
      \Finp\op
      \arrow[r, hook]
      & \Map(\Delta^{1}, \Finp\op)
    \end{tikzcd}.
  \end{equation*}
  By \cite[Prop.\ 3.1.2.1]{highertopostheory}, the map $q'$ is a cartesian fibration whose cartesian morphisms are pointwise cartesian morphisms in $\S_{\times}$. The map $q$ is also a cartesian fibration, where a morphism in $\Map(\Delta^{1}, \S)_{\times}$ is $q$-cartesian if and only if its image in $\Map(\Delta^{1}, \S_{\times})$ is $q'$-cartesian. That every $q$ cartesian morphism is $q'$ cartesian follows from \cite[Prop\ 2.4.1.3]{highertopostheory}; the converse follows from the existence of comparison equivalences between cartesian lifts of the same morphism.
\end{proof}

\begin{lemma}
  \label{lemma:map_on_cartesian_cats_is_bicart_fib}
  The map $r$ is a bicartesian fibration with $r$-cartesian morphisms pullback squares in $\S_{\times}$ and $r$-cocartesian morphisms squares of the form
  \begin{equation*}
    \begin{tikzcd}
      X
      \arrow[r, "\simeq"]
      \arrow[d]
      & Y
      \arrow[d]
      \\
      X'
      \arrow[r]
      & Y'
    \end{tikzcd}.
  \end{equation*}
\end{lemma}
\begin{proof}
  The map $r$ factors into
  \begin{equation*}
    \begin{tikzcd}
      \Map(\Delta^{1}, \S)_{\times}
      \arrow[r, hook, "j"]
      & \Map(\Delta^{1}, \S_{\times})
      \arrow[r, "\ev_{1}"]
      & \S_{\times}
    \end{tikzcd}
  \end{equation*}
  The map $\ev_{1}$ is a bicartesian fibration with the cartesian and cocartesian morphisms described above; the map $j$ is a full inclusion (in the sense of \hyperref[def:full_inclusion]{Definition~\ref*{def:full_inclusion}}), and therefore has the right-lifting property with respect to \emph{all} horn inclusions $\Lambda^{n}_{i} \hookrightarrow \Delta^{n}$, $n \geq 2$ (because $\Lambda^{n}_{i}$ contains all vertices for $n \geq 2$). Every edge in $\Map(\Delta^{1}, \S)_{\times}$ is therefore both $j$-cartesian and $j$-cocartesian. \cite[Prop\ 2.4.1.3.3]{highertopostheory} implies that an edge in $\Map(\Delta^{1}, \S)_{\times}$ is $r$-cartesian (resp.\ cocartesian) if and only if its image in $\Map(\Delta^{1}, \S_{\times})$ is $\ev_{1}$-cartesian (resp.\ cocartesian).
\end{proof}

Let us take stock. We have now defined the data of the commutative diagram below. Here, $p$ and $q$ are cartesian fibrations, and $r$ is a bicartesian fibration.
\begin{equation}
  \label{eq:data_of_pqr}
  \begin{tikzcd}
    \Map(\Delta^{1}, \S)_{\times}
    \arrow[rr, "r"]
    \arrow[dr, swap, "q"]
    && \S_{\times}
    \arrow[dl, "p"]
    \\
    & \Finp\op
  \end{tikzcd}
\end{equation}
Our next step is to take spans with judiciously-chosen legs in each of these quasicategories. To do this, we must define the subcategories to which the legs of our spans are allowed to belong. As in \hyperref[sec:segal_spaces_of_spans]{Section~\ref*{sec:segal_spaces_of_spans}}, we do this by defining triples of categories.

\subsection{Defining the adequate triple structures}
\label{ssc:restricting_the_legs_of_the_spans}

In this section, we continue our construction of the functor $\hat{r}$, following the plan laid out in \hyperref[sss:sketch_of_lax_monoidal_structure]{Subsubsection~\ref*{sss:sketch_of_lax_monoidal_structure}}. Our next step is to define triple structures on the quasicategories $\Map(\Delta^{1}, \S)_{\times}$, $\S_{\times}$, and $\Finp\op$, which specify the categories of spans in \hyperref[eq:wishlist_pivarpirho]{Diagram~\ref*{eq:wishlist_pivarpirho}}.

\subsubsection{Defining the triple structure on \texorpdfstring{$\Finp\op$}{Fin.op}}

Recall \hyperref[ssc:introduction]{Subsection~\ref*{ssc:introduction}}. We have cartesian fibrations $p$, $q$, and $r$ as in \hyperref[eq:data_of_pqr]{Diagram~\ref*{eq:data_of_pqr}}; these play the role of the maps alluded to in \hyperref[eq:wishlist_prq]{Diagram~\ref*{eq:wishlist_prq}}. We would like a triple structure on $\Finp\op$ such that, when passing to quasicategories of spans, we find cocartesian fibrations of the form \hyperref[eq:wishlist_pivarpirho]{Diagram~\ref*{eq:wishlist_pivarpirho}}.

This means, in particular, that we want a triple structure on $\Finp\op$ which gives us something equivalent to $\Finp$ upon taking categories of spans. Thus, we should take spans whose forward-pointing legs are equivalences:
\begin{equation*}
  \begin{tikzcd}
    \langle m \rangle
    \arrow[r]
    & \langle n \rangle
    & \langle n \rangle
    \arrow[l, swap, "\simeq"]
  \end{tikzcd}.
\end{equation*}

We have considered this situation before, in \hyperref[eg:spans_with_equivalences_on_forwards_leg]{Example~\ref*{eg:spans_with_equivalences_on_forwards_leg}}. There, we saw that by considering spans in some quasicategory $\category{C}$ whose forward-facing legs were equivalences, we found a categorical equivalence $\category{C}\op \to \Span_{\simeq}(\category{C})$.

Thus, we will take on $\Finp\op$ the triple structure defined in \hyperref[eg:spans_with_equivalences_on_forwards_leg]{Example~\ref*{eg:spans_with_equivalences_on_forwards_leg}}.

\begin{definition}
  We define a triple $\triple{F}$ as follows.
  \begin{itemize}
    \item $\mathcal{F} = \Finp\op$

    \item $\mathcal{F}\downdag = (\Finp\op)^{\simeq}$

    \item $\mathcal{F}\updag = \Finp\op$
  \end{itemize}
\end{definition}

For posterity, we record the content of \hyperref[eg:spans_with_equivalences_on_forwards_leg]{Example~\ref*{eg:spans_with_equivalences_on_forwards_leg}} in the following proposition.

\begin{lemma}
  There is a weak Joyal equvialence
  \begin{equation*}
    \Finp \to \Span\triple{F},
  \end{equation*}
  which sends an $n$-simplex $\langle m \rangle \to \langle n \rangle \to \cdots \to \langle p \rangle$ to a diagram
  \begin{equation*}
    \begin{tikzcd}[row sep=small, column sep=tiny]
      &&& \respace{\langle p \rangle}
      \\
      && \respace{\iddots}
      \arrow[ur]
      && \respace{\ddots}
      \arrow[ul, equals]
      \\
      & \respace{\langle n \rangle}
      \arrow[ur]
      && \respace{\iddots}
      \arrow[ur]
      && \respace{\langle p \rangle}
      \arrow[ul, equals]
      \\
      \langle m \rangle
      \arrow[ur]
      && \respace{\langle n \rangle}
      \arrow[ul, equals]
      \arrow[ur]
      && \respace{\dots}
      && \respace{\langle p \rangle}
      \arrow[ul, equals]
    \end{tikzcd},
  \end{equation*}
  where we have drawn the morphisms in the span as belonging to $\Finp$ rather than $\Finp\op$.
\end{lemma}

\subsubsection{Defining the triple structure on \texorpdfstring{$\S_{\times}$}{S}}

Next, we define a triple structure on $\S_{\times}$.

\begin{definition}
  We define a triple $\triple{P}$ as follows.
  \begin{itemize}
    \item $\mathcal{P} = \mathcal{S}_{\times}$

    \item $\mathcal{P}\downdag = \mathcal{S}_{\times} \times_{\Finp\op} (\Finp\op)^{\simeq}$

    \item $\mathcal{P}\updag = \mathcal{S}_{\times}$
  \end{itemize}
\end{definition}

Here, we do not restrict the legs of our spans at all, except to ensure that $p$ restricts to a map $p\downdag\colon \category{P}\downdag \to \category{F}\downdag$.

Next, we need to show that the triple $\triple{P}$ is adequate. For this we need a few lemmas.

\begin{lemma}
  \label{lemma:spaces_fibration_admits_relative_pullbacks}
  The cartesian fibration $p\colon \S_{\times} \to \Finp\op$ admits relative pullbacks.
\end{lemma}
\begin{proof}
  By \cite[Cor.\ 4.3.1.11]{highertopostheory}, it suffices to show that each fiber $(\S_{\times})_{\langle n \rangle}$ admits pullbacks, and that for each map $\phi\colon \langle m \rangle \leftarrow \langle n \rangle$ in $\Finp\op$, the map $\phi^{*}\colon (\S_{\times})_{\langle m \rangle} \to (\S_{\times})_{\langle n \rangle}$ preserves pullbacks. By the defintion of the cocartesian monoidal structure on $\S\op$, we have an isomorphism  $(\S_{\times})_{\langle n \rangle} \cong \S^{n}$, and $\S^{n}$ admits pullbacks. Hence, each fiber admits pullbacks.

  Now we show that for any map $\phi\colon \langle m \rangle \leftarrow \langle n \rangle$ in $\Finp\op$, the map $\phi^{*}$ preserves pullbacks. Unravelling the definitions, we find that $\phi^{*}$ sends an object $(X_{1}, \ldots, X_{m}) \in \left( \S_{\times} \right)_{\langle m \rangle}$ to an object $(Y_{1}, \ldots, Y_{n}) \in (\S_{\times})_{\langle n \rangle}$ such that
  \begin{equation*}
    X_{i} \simeq \prod_{\phi(j) = i} Y_{j},
  \end{equation*}
  and the action on higher simplices is determined by the universal property for products. Since pullbacks in $\S^{m}$ are level-wise pullbacks in $\S$ and products commute with pullbacks, the result follows.
\end{proof}

\begin{lemma}
  \label{lemma:existence_and_preservation_of_limits}
  Let $p\colon \category{C} \to \category{D}$ be a cartesian fibration between quasicategories, and let $K$ be a simplicial set. Suppose that $\category{D}$ admits all $K$-shaped limits, and that $p$ admits all relative $K$-shaped limits. Then the following hold.
  \begin{enumerate}
    \item The quasicategory $\category{C}$ admits all $K$-shaped limits.

    \item The map $p$ preserves all $K$-shaped limits.
  \end{enumerate}
\end{lemma}
\begin{proof}
  \begin{enumerate}
    \item For any diagram $q\colon K \to \category{C}$, the diagram $p \circ q\colon K \to \category{D}$ admits a limit cone $\overline{p \circ q}\colon K\cone \to \category{D}$, and this data admits a relative $p$-limit cone $\bar{q}$ as below.
      \begin{equation*}
        \begin{tikzcd}
          K
          \arrow[r, "q"]
          \arrow[d, hook]
          & \category{C}
          \arrow[d, "p"]
          \\
          K\cone
          \arrow[r, swap, "\overline{p \circ q}"]
          \arrow[ur, dashed, "\bar{q}"]
          & \category{D}
        \end{tikzcd}
      \end{equation*}
      This is a limit cone for $q$ by \cite[Prop.\ 4.3.1.5]{highertopostheory}.

    \item Let $\bar{q}\colon K\cone \to \category{C}$ be a limit cone, and write $q = \bar{q}|_{K}$. We need to show that $p \circ \bar{q}\colon K\cone \to \category{D}$ is a limit cone in $\category{D}$. The restriction $p \circ q\colon K \to \category{D}$ admits a limit cone $\widehat{p \circ q}\colon K\cone \to \category{D}$, which admits a relative lift to a limit cone $\hat{q}\colon K\cone \to \category{C}$. But $\bar{q}$ and $\hat{q}$ are both limit cones of $q$, hence are equivalent as objects of $\Map(K\cone, \category{C})$ (by \cite[Prop.\ 4.3.1.5]{highertopostheory}). Therefore, $\widehat{p \circ q} = p \circ \hat{q}$ and $p \circ \bar{q}$ are equivalent as objects of $\Map(K\cone, \category{D})$, so $p \circ \bar{q}$ is a limit cone since $\widehat{p \circ q}$ is.
  \end{enumerate}
\end{proof}

\begin{proposition}
  The triple $\triple{P}$ is adequate.
\end{proposition}
\begin{proof}
  We check the conditions of \hyperref[def:adequate_triple]{Definition~\ref*{def:adequate_triple}}.
  \begin{enumerate}
    \item The category $\S_{\times}$ admits all pullbacks by \hyperref[lemma:spaces_fibration_admits_relative_pullbacks]{Lemma~\ref*{lemma:spaces_fibration_admits_relative_pullbacks}}, hence certainly admits ambigressive pullbacks.

    \item We need to show that for any pullback square in $\S_{\times}$
      \begin{equation*}
        \begin{tikzcd}
          X
          \arrow[r]
          \arrow[d, swap, "f"]
          & X'
          \arrow[d, "f'"]
          \\
          Y
          \arrow[r]
          & Y'
        \end{tikzcd}
      \end{equation*}
      such that $f$ lies over an isomorphism in $\Finp\op$, so does $f'$. But \hyperref[lemma:existence_and_preservation_of_limits]{Lemma~\ref*{lemma:existence_and_preservation_of_limits}} implies that the above square lies over a pullback square in $\Finp\op$, and the pullback of an isomorphism is an isomorphism.
  \end{enumerate}
\end{proof}

\subsubsection{Defining the triple structure on \texorpdfstring{$\Map(\Delta^{1}, \S)_{\times}$}{Map}}

Lastly, we define a triple structure on $\Map(\Delta^{1}, \S)_{\times}$.
\begin{definition}
  We define a triple $\triple{Q}$ as follows.
  \begin{itemize}
    \item $\mathcal{Q} = \Map(\Delta^{1}, \S)_{\times}$.

    \item $\mathcal{Q}\downdag$ is given by the pullback
      \begin{equation*}
        \begin{tikzcd}
          \mathcal{Q}\downdag
          \arrow[r]
          \arrow[d]
          & \mathcal{Q}
          \arrow[d, "{(q, \ev_{0})}"]
          \\
          (\Finp\op)^{\simeq} \times \S_{\times}^{\simeq}
          \arrow[r]
          & \Finp\op \times \S_{\times}
        \end{tikzcd}.
      \end{equation*}

    \item $\mathcal{Q}\updag = \mathcal{Q}$.
  \end{itemize}
\end{definition}

\begin{lemma}
  The functor $q\colon \Map(\Delta^{1}, \S)_{\times} \to \Finp\op$ admits relative pullbacks.
\end{lemma}
\begin{proof}
  We need to show that each fiber $(\Map(\Delta^{1}, \S)_{\times})_{\langle n \rangle}$ admits pullbacks. Using the isomorphisms $(\S_{\times})_{\langle n \rangle} \simeq \S^{n}$, we find isomorphisms
  \begin{equation*}
    (\Map(\Delta^{1}, \S)_{\times})_{\langle n \rangle} \cong \Map(\Delta^{1}, \S)^{n}.
  \end{equation*}
  The proof of \hyperref[lemma:spaces_fibration_admits_relative_pullbacks]{Lemma~\ref*{lemma:spaces_fibration_admits_relative_pullbacks}} then applies to our situation, with $\S$ replaced by $\Map(\Delta^{1}, \S)$.
\end{proof}

\begin{lemma}
  The triple $\triple{Q}$ is adequate.
\end{lemma}
\begin{proof}
  We check the conditions of \hyperref[def:adequate_triple]{Definition~\ref*{def:adequate_triple}}.
  \begin{enumerate}
    \item Since $q$ admits relative pullbacks and $\Finp\op$ admits pullbacks, \hyperref[lemma:existence_and_preservation_of_limits]{Lemma~\ref*{lemma:existence_and_preservation_of_limits}} implies that $\Map(\Delta^{1}, \S)_{\times}$ admits all pullbacks, hence certainly ambigressive pullbacks.

    \item Consider a pullback square in $\Map(\Delta^{1}, \S)_{\times}$ corresponding to a cube in $\S_{\times}$
      \begin{equation}
        \label{eq:cube_in_Q}
        \begin{tikzcd}
          S'
          \arrow[rr, "F"]
          \arrow[dd]
          \arrow[dr]
          && T'
          \arrow[dd]
          \arrow[dr]
          \\
          & S
          \arrow[rr, crossing over, near start, "f"]
          && T
          \arrow[dd]
          \\
          X'
          \arrow[dr]
          \arrow[rr]
          && Y'
          \arrow[dr]
          \\
          & X
          \arrow[from=uu, crossing over]
          \arrow[rr]
          && Y
        \end{tikzcd},
      \end{equation}
      and which is lying over a square
      \begin{equation}
        \label{eq:square_in_finp_under_Q}
        \begin{tikzcd}
          \langle n \rangle
          & \langle n' \rangle
          \arrow[l, swap, "\psi"]
          \\
          \langle m \rangle
          \arrow[u]
          & \langle m' \rangle
          \arrow[l, swap, "\phi"]
          \arrow[u]
        \end{tikzcd}
      \end{equation}
      in $\Finp\op$. We need to show that (a) if $f$ is an equivalence, then $F$ is an equivalence, and (b) if $\phi$ is an equivalence, then $\psi$ is an equivalence.
      \begin{enumerate}
        \item By \cite[Cor.\ 4.3.1.15]{highertopostheory} with $\category{C} = \Map(\Delta^{1}, \S_{\times})$, $\category{D} = \category{E} = \Map(\Delta^{1}, \Finp\op)$, and $\category{E}_{0} = \Finp\op$, the cube in \hyperref[eq:cube_in_Q]{Diagram~\ref*{eq:cube_in_Q}} corresponds to a pullback diagram in $\Map(\Delta^{1}, \S)_{\times}$ if and only if it corresponds to a pullback square in $\Map(\Delta^{1}, \S_{\times})$, which in turn is true if and only if the top and bottom squares are pullback in $\S_{\times}$. The result follows because pullbacks preserve equivalences.

        \item \hyperref[lemma:existence_and_preservation_of_limits]{Lemma~\ref*{lemma:existence_and_preservation_of_limits}} implies that the square in \hyperref[eq:square_in_finp_under_Q]{Diagram~\ref*{eq:square_in_finp_under_Q}} is pullback (in $\Finp\op$), and the result again follows because pullbacks preserve equivalences.
      \end{enumerate}
  \end{enumerate}
\end{proof}

As we have defined them, it is clear that $p$, $q$, and $r$ restrict to maps
\begin{equation*}
  \begin{tikzcd}[column sep=small]
    \mathcal{M}_{\dagger}
    \arrow[r, "r_{\dagger}"]
    \arrow[rr, swap,  bend right, "q_{\dagger}"]
    & \mathcal{P}_{\dagger}
    \arrow[r, "q_{\dagger}"]
    & \mathcal{F}_{\dagger}
  \end{tikzcd}
  \qquad\text{and}\qquad
  \begin{tikzcd}[column sep=small]
    \mathcal{M}^{\dagger}
    \arrow[r, "r^{\dagger}"]
    \arrow[rr, swap,  bend right, "q^{\dagger}"]
    & \mathcal{P}^{\dagger}
    \arrow[r, "q^{\dagger}"]
    & \mathcal{F}^{\dagger}
  \end{tikzcd},
\end{equation*}
and preserve ambigressive pullback squares.


\subsection{Building categories of spans}
\label{ssc:building_categories_of_spans}

We have now constructued the following diagram of adequate triples and maps between them.
\begin{equation*}
  \begin{tikzcd}
    \triple{Q}
    \arrow[rr, "r"]
    \arrow[dr, swap, "q"]
    && \triple{P}
    \arrow[dl, "p"]
    \\
    & \triple{F}
  \end{tikzcd}
\end{equation*}

This gives us a commuting triangle
\begin{equation}
  \label{eq:commuting_diagram_of_spans}
  \begin{tikzcd}
    \Span\triple{Q}
    \arrow[rr, "\rho'"]
    \arrow[dr, swap, "\varpi'"]
    && \Span\triple{P}
    \arrow[dl, "\pi'"]
    \\
    & \Span\triple{F}
  \end{tikzcd}.
\end{equation}
Our next step is to show that the maps $\pi'$, $\rho'$, and $\varpi'$ are cocartesian fibrations. To do this, we need to show that they are inner fibrations, and that they admit enough cocartesian lifts; both of these will be assured by \hyperref[thm:main]{Theorem~\ref*{thm:main}}, assuming we can show that this theorem applies.

\begin{lemma}
  The morphism $p$ satisfies the conditions of \hyperref[thm:main]{Theorem~\ref*{thm:main}}.
\end{lemma}
\begin{proof}
  We check the conditions.
  \begin{enumerate}
    \item We have to show that for any isomorphism in $\Finp\op$, there exists a $p$-cocartesian lift. But since every morphism in $\mathcal{F}_{\dagger} = (\Finp\op)^{\simeq}$ is an equivalence, it suffices to find a $p$-cartesian lift, which is possible because $p$ is a cartesian fibration.

    \item We need to show that for any commutative square
      \begin{equation*}
        \sigma \ = \quad
        \begin{tikzcd}
          S'
          \arrow[r, "f"]
          \arrow[d]
          & T'
          \arrow[d]
          \\
          S
          \arrow[r, "\simeq"]
          & T
        \end{tikzcd}
      \end{equation*}
      lying over a square
      \begin{equation*}
        \begin{tikzcd}
          \langle n \rangle
          & \langle n \rangle
          \arrow[l, swap, "\cong"]
          \\
          \langle m \rangle
          \arrow[u]
          & \langle m \rangle
          \arrow[l, swap, "\cong"]
          \arrow[u]
        \end{tikzcd}
      \end{equation*}
      in $\Finp\op$ (where we have drawn the arrows as belonging to $\Finp$), $\sigma$ is a pullback square if and only if $f$ is $p$-cocartesian. But $f$ is lying over an equivalence, so it is $p$-cocartesian if and only if it is an equivalence, which in turn is true if and only if the square in question is pullback.
  \end{enumerate}
\end{proof}

\begin{lemma}
  The morphism $q$ satisfies the conditions of \hyperref[thm:main]{Theorem~\ref*{thm:main}}.
\end{lemma}
\begin{proof}
  We have to check the conditions of the \hyperref[thm:main]{Theorem~\ref*{thm:main}}:
  \begin{enumerate}
    \item We have to show that for any isomorphism $\phi$ in $\Finp\op$, there exists a $q$-cocartesian lift. It follows from the $2/3$ property for equivalences that such a lift will automatically be $q\downdag$-cocartesian. But because $\phi$ is an equivalence, it suffices to find a $q$-cartesian lift, which is possible because $q$ is a cartesian fibration.

    \item We need to show that for any commutative square in $\Map(\Delta^{1}, \S)_{\times}$ corresponding to a cube in $\S_{\times}$ of the form
      \begin{equation*}
        \begin{tikzcd}
          X'
          \arrow[rr, "F"]
          \arrow[dd]
          \arrow[dr]
          && Y'
          \arrow[dd]
          \arrow[dr]
          \\
          & X
          \arrow[rr, crossing over, near start, "\simeq"]
          && Y
          \arrow[dd]
          \\
          S'
          \arrow[dr]
          \arrow[rr, near start, "f"]
          && T'
          \arrow[dr]
          \\
          & S
          \arrow[from=uu, crossing over]
          \arrow[rr, "\simeq"]
          && T
        \end{tikzcd}
      \end{equation*}
      and lying over a square in $\Finp\op$ of the form
      \begin{equation*}
        \begin{tikzcd}
          \langle n \rangle
          & \langle n \rangle
          \arrow[l, swap, "\cong"]
          \\
          \langle m \rangle
          \arrow[u]
          & \langle m \rangle
          \arrow[l, swap, "\cong"]
          \arrow[u]
        \end{tikzcd},
      \end{equation*}
      the morphism $(F, f)$ is $p$-cocartesian if and only if the top and bottom squares are pullback and $\psi$ is $p$-cocartesian. But this again follows because equivalences are $p$-cocartesian, and cocartesian morphisms are closed under composition.
  \end{enumerate}
\end{proof}

\begin{lemma}
  The map $r$ satisfies the conditions of \hyperref[thm:main]{Theorem~\ref*{thm:main}}.
\end{lemma}
\begin{proof}
  \begin{enumerate}
    \item It suffices to show that for each morphism $T \to S'$ in $\S_{\times}$ lying over $\langle m \rangle \overset{\cong}{\leftarrow} \langle m \rangle$ and each $Y \to T$ in the fiber over $\langle m \rangle$, the square
      \begin{equation*}
        \begin{tikzcd}
          Y
          \arrow[r, equals]
          \arrow[d]
          & Y
          \arrow[d]
          \\
          T
          \arrow[r]
          & S'
        \end{tikzcd}
      \end{equation*}
      is both $p$-cocartesian and $p_{\dagger}$-cocartesian, both of which follow from \hyperref[lemma:map_on_cartesian_cats_is_bicart_fib]{Lemma~\ref*{lemma:map_on_cartesian_cats_is_bicart_fib}}.

    \item We need to show that for any commutative square in $\Map(\Delta^{1}, \S)_{\times}$ corresponding to a cube in $\S_{\times}$ of the form
      \begin{equation*}
        \begin{tikzcd}
          X'
          \arrow[rr, "F"]
          \arrow[dd]
          \arrow[dr]
          && Y'
          \arrow[dd]
          \arrow[dr]
          \\
          & X
          \arrow[rr, crossing over, near start, "\simeq"]
          && Y
          \arrow[dd]
          \\
          S'
          \arrow[dr]
          \arrow[rr, near start, "f"]
          && T'
          \arrow[dr]
          \\
          & S
          \arrow[from=uu, crossing over]
          \arrow[rr]
          && T
        \end{tikzcd}
      \end{equation*}
      where the bottom square is pullback, the morphism $(F, f)$ is $r$-cocartesian if and only if the top square is pullback and $\psi$ is an equivalence. This is clear.
  \end{enumerate}
\end{proof}

\hyperref[thm:main]{Theorem~\ref*{thm:main}} then tells us immediately that the maps in \hyperref[eq:commuting_diagram_of_spans]{Diagram~\ref*{eq:commuting_diagram_of_spans}} are inner fibrations. Thus, in order to show that they are cocartesian fibrations, it suffices to exhibit sufficient cocartesian lifts.

\begin{proposition}
  The maps $\pi'$, $\varpi'$, and $\rho'$ are cocartesian fibrations.
\end{proposition}
\begin{proof}
  We need to check that each of $\pi'$, $\varpi'$, and $\rho'$ admits enough cocartesian lifts. \hyperref[thm:main]{Theorem~\ref*{thm:main}} provides a sufficient condition that morphisms be cocartesian, so we can simply check that lifts of these forms exist. Fix be a morphism
  \begin{equation*}
    \phi =
    \begin{tikzcd}
      \langle m \rangle
      \arrow[r]
      & \langle n \rangle
      & \langle n \rangle
      \arrow[l, swap, "\simeq"]
    \end{tikzcd}
  \end{equation*}
  in $\Span\triple{F}$, and a morphism
  \begin{equation*}
    \Gamma =
    \begin{tikzcd}
      S
      & T
      \arrow[l]
      \arrow[r]
      & S'
    \end{tikzcd}
  \end{equation*}
  in $\Span\triple{P}$
  \begin{itemize}
    \item Given an object $X$ in $\Span\triple{P}$ lying over $\langle m \rangle$, \hyperref[thm:main]{Theorem~\ref*{thm:main}} tells us that a $\pi'$-cocartesian lift of $\phi$ is of the form
      \begin{equation*}
        \begin{tikzcd}
          X
          & Y
          \arrow[l, swap, "f"]
          \arrow[r, "\simeq"]
          & X'
        \end{tikzcd},
      \end{equation*}
      where $f$ is $p$-cartesian. Such lifts always exist because $p$ is a cartesian fibration.

    \item Given an object $S \to X$ in $\Span\triple{Q}$ lying over $\langle m \rangle$, a $\varpi'$-cocartesian lift of $\phi$ is of the form
      \begin{equation*}
        \begin{tikzcd}
          S
          \arrow[d]
          & T
          \arrow[l, swap, "F"]
          \arrow[r, "\simeq"]
          \arrow[d]
          & S'
          \arrow[d]
          \\
          X
          & Y
          \arrow[l, swap, "f"]
          \arrow[r, "\simeq"]
          & X'
        \end{tikzcd},
      \end{equation*}
      where $F$ and $f$ are $p$-cartesian (i.e.\ the left-hand square corresponds to a $q$-cartesian morphism in $\Map(\Delta^{1}, \S)_{\times}$). Such lifts automatically exist because $q$ is a cartesian fibration.

    \item Given an object $S \to X$ in $\Span\triple{Q}$ lying over $\langle m \rangle$, a $\rho'$-cocartesian lift of $\phi$ is of the form
      \begin{equation*}
        \begin{tikzcd}
          S
          \arrow[d]
          & S \times_{X} Y
          \arrow[l]
          \arrow[r, "\simeq"]
          \arrow[d]
          & S'
          \arrow[d]
          \\
          X
          & Y
          \arrow[l]
          \arrow[r]
          & X'
        \end{tikzcd},
      \end{equation*}
      where the left-hand square is pullback. Such lifts exist because $\S_{\times}$ admits pullbacks.
  \end{itemize}
\end{proof}

Recall that there is an equivalence $\Finp \to \Span\triple{F}$. Pulling back \hyperref[eq:commuting_diagram_of_spans]{Diagram~\ref*{eq:commuting_diagram_of_spans}} along this map gives a diagram of cocartesian fibrations
\begin{equation*}
  \begin{tikzcd}
    P
    \arrow[rr, "\rho"]
    \arrow[dr, swap, "\varpi"]
    && Q
    \arrow[dl, "\pi"]
    \\
    & \Finp
  \end{tikzcd}.
\end{equation*}

\begin{proposition}
  The maps $\pi$ and $\varpi$ exhibit $P$ and $Q$ as symmetric monoidal categories in the sense of \cite[Def.~2.0.0.7]{luriehigheralgebra}.
\end{proposition}
\begin{proof}
  We consider the case of $\pi$. We need to show that the pushforward functors corresponding to the inert maps $\rho_{i}\colon \langle n \rangle \to \langle 1 \rangle$ induce an equivalence $P_{\langle n \rangle} \simeq P_{\langle 1 \rangle}^{n}$. These maps send (up to equivalence) $[X_{1}, \ldots, X_{n}] \mapsto X_{i}$, giving an isomorphism $P_{\langle n \rangle} \cong P_{\langle 1 \rangle}^{n}$.

  The case of $\varpi$ is similar.
\end{proof}

\subsection{Building the functor \texorpdfstring{$\hat{r}$}{r}}
\label{ssc:building_the_functor}

We are now ready to build the lax monoidal functor $\hat{r}\colon \Span(\S) \to \ICat$. Restricting the functor $\rho$ to the component lying over $\langle 1 \rangle \in \Finp$ gives a cocartesian fibration $\rho_{\langle 1 \rangle\colon }Q_{\langle 1 \rangle} \to P_{\langle 1 \rangle}$, which classifies a functor $P_{\langle 1 \rangle} \to \ICat$. This is the functor described in \hyperref[sss:sketch_of_underlying_functor]{Subsubsection~\ref*{sss:sketch_of_underlying_functor}}. This will be almost the functor we need, as the following results show.

\begin{proposition}
  There is an equivalence of simplicial sets $P_{\langle 1 \rangle} \simeq \Span(\S)$.
\end{proposition}
\begin{proof}
  Unravelling the defintitions, the $n$-simplices of $P_{\langle 1 \rangle}$ are ambigressive cartesian maps $\sigma\colon \sd(\Delta^{n}) \to \S_{\times}$ which make the diagram
  \begin{equation*}
    \begin{tikzcd}
      \sd(\Delta^{n})
      \arrow[rr, "\sigma"]
      \arrow[dr, swap, "\const_{\langle 1 \rangle}"]
      && \S_{\times}
      \arrow[dl]
      \\
      & \Finp\op
    \end{tikzcd}
  \end{equation*}
  commute. The condition that the diagram commute is the condition that $\sigma$ land in $(\S_{\times})_{\langle 1 \rangle} \simeq \S$, and the ambigressiveness condition is that the forward-facing legs in $\sigma$ lie over equivalences in $\Finp\op$, which is automatic. This provides an isomorphism of simplicial sets $P_{\langle 1 \rangle} \cong \Span(\S)$, which is stronger than what we are looking for.
\end{proof}

\begin{proposition}
  For any $X \in P_{\langle 1 \rangle}$, there is an equivalence
  \begin{equation*}
    (Q_{\langle 1 \rangle})_{X} \simeq (\S_{/X})\op.
  \end{equation*}
\end{proposition}
\begin{proof}
  The $n$-simplices of $(Q_{\langle 1 \rangle})_{X}$ are ambigressive cartesian maps $\sigma\colon \sd(\Delta^{n}) \to \Map(\Delta^{1}, \S)_{\times}$ such that the diagrams
  \begin{equation*}
    \begin{tikzcd}
      \sd(\Delta^{n})
      \arrow[rr, "\sigma"]
      \arrow[dr, swap, "\const_{\langle 1 \rangle}"]
      && \Map(\Delta^{1}, \S)_{\times}
      \arrow[dl]
      \\
      & \Finp\op
    \end{tikzcd}
  \end{equation*}
  and
  \begin{equation*}
    \begin{tikzcd}
      \sd(\Delta^{n})
      \arrow[rr, bend left, "\const_{X}"]
      \arrow[r, "\sigma"]
      & \Map(\Delta^{1}, \S)_{\times}
      \arrow[r, "\ev_{1}"]
      & \S_{\times}
    \end{tikzcd}
  \end{equation*}
  commute. The first diagram ensures that $\sigma$ lands entirely inside the simplicial subset $(\Map(\Delta^{1}, \S)_{\times})_{\langle 1 \rangle} \cong \Map(\Delta^{1}, \S)$, so any such $\sigma$ is uniquely determined by a map $\sd(\Delta^{n}) \times \Delta^{1} \to \S$ such that:
  \begin{itemize}
    \item The composition
      \begin{equation*}
        \sd(\Delta^{n}) \times \Delta^{\{1\}} \hookrightarrow \sd(\Delta^{n}) \times \Delta^{1} \to \S
      \end{equation*}
      is equal to $\const_{X}$.

    \item The forward-facing legs in $\sd(\Delta^{n}) \times \Delta^{\{0\}}$ are mapped to equivalences in $\S$.
  \end{itemize}
  This is precisely the data of a map $\sd(\Delta^{n}) \diamond \Delta^{0} \to \S$ sending $\Delta^{0} \mapsto X$, and such that and the forward-facing legs of $\sd(\Delta^{n})$ are sent to equivalences in $\S$. But these are precisely the $n$-simplices of $\Span_{\simeq}(\S^{/X})$. This provides an isomorphism $Q_{\langle 1 \rangle} \cong \Span_{\simeq}(\S^{/X})$. The equvialences $(\S^{/X})\op \to \Span_{\simeq}(\S^{/X})$ and $\S_{/X} \simeq \S^{/X}$ now give us the equivalence that we want.
\end{proof}

We now define the functor 
\begin{equation*}
  R\colon \Span(\S) \to \ICat
\end{equation*}
to be the functor classified by the cocartesian fibration $\rho_{\langle 1 \rangle}$. On objects, this functor sends a space $X$ to the category $(\S_{/X})\op$. A morphism in $\Span(\S)$ represented by a span of spaces
\begin{equation*}
  \begin{tikzcd}
    X
    & Y
    \arrow[l, swap, "g"]
    \arrow[r, "f"]
    & X'
  \end{tikzcd}
\end{equation*}
is sent to the functor $(\S_{/X})\op \to (\S_{/X'})\op$ defined by sending an object $S \to X$ to the object $S' \to X'$ defined up to equivalence as follows.
\begin{equation*}
  \begin{tikzcd}
    S
    \arrow[d]
    & S \times_{X} Y
    \arrow[l]
    \arrow[r, "\simeq"]
    \arrow[d]
    & S'
    \arrow[d]
    \\
    X
    & Y
    \arrow[l, swap, "g"]
    \arrow[r, "f"]
    & X'
  \end{tikzcd}
\end{equation*}
This is almost what we want, up to an $\mathrm{op}$: we would like our functor $\hat{r}$ to send $X \mapsto \S_{/X}$ rather than $(\S_{/X})\op$. Fortunately, this is only an inconvenience. Recall that there is an autoequivalence of $\ICat$ sending $\category{C} \mapsto \category{C}\op$.
\begin{definition}
  We define the map $\hat{r}\colon \Span(\S) \to \ICat$ to be the composition
  \begin{equation*}
    \begin{tikzcd}
      \Span(\S)
      \arrow[r, "R"]
      & \ICat
      \arrow[r, "\mathrm{op}"]
      & \ICat
    \end{tikzcd}.
  \end{equation*}
\end{definition}

The map $R$ admits a lax monoidal structure by \cite[2.4.2.4--2.4.2.6]{luriehigheralgebra}, and the map $\mathrm{op}$ is an equivalence, hence certainly monoidal with respect to the cartesian monoidal structure on $\ICat$. Thus, $\hat{r}$ is lax monoidal.

\subsection{Conclusion and outlook}
\label{ssc:conclusion}

Let us meditate on our construction of the functor $\hat{r}$. Our starting point was the fact that the assignment of a space $X$ to the quasicateogry $\P(X)$ of presheaves on $X$ admits both covariant and contravariant functoriality. More specifically, we saw that for any map of spaces $f\colon X \to Y$, we could construct the following maps of presheaves.
\begin{itemize}
  \item The left Kan extension functor $f_{!}\colon \P(X) \to \P(Y)$,

  \item The pullback functor $f^{*}\colon \P(Y) \to \P(X)$.
\end{itemize}
The existence of these two maps in opposite directions was the key ingredient that allowed us to construct our map $\hat{r}\colon \Span(\S) \to \ICat$.

There are several immediate avenues of generalization which could be explored. For example, one could replace $\S$-valued presheaves by presheaves valued in some other $\infty$-category $\category{C}$, or replace presheaves by sheaves. In both of these situations one has analogs of the pullback and pushforward functors.

However, as we alluded to in \hyperref[sec:introduction]{Section~\ref*{sec:introduction}}, having a way of producing from a morphism in some category $\category{C}$ a morphism in some category $\category{D}$ in both a covariant and a contravariant way is a rather common situation. The following are examples.

\begin{itemize}
  \item Let $p\colon X \to Y$ be a covering space map with finite fibers, say of cardinality $n$. This induces a map $p_{*}\colon H_{*}X \to H_{*}Y$ on singular homology, as does any continuous map of topological spaces. Additionally, for any singular $n$-simplex $\sigma\colon |\Delta^{n}| \to Y$, the homotopy lifting property allows us to construct a singular $n$-chain $\alpha$ on $X$ such that $p_{*}(\alpha) = n$. Roughly, $\alpha$ is a sum of singular $n$-simplices in $X$ each lying above $\sigma$, one in each fiber. This extends to a map of singular chain complexes, and descends to a transfer map $t\colon H_{*}Y \to H_{*}X$.

  \item Given a covering space map with finite fibers $p\colon X \to Y$ as above, Roush showed in \cite{roushtransfer} that one can construct a stable transfer map $\Sigma^{\infty} Y_{+} \to \Sigma^{\infty} X_{+}$.

  \item Becker and Gottlieb generalized this result in \cite{beckergottlieb} to the case of a fiber bundle $E \to B$ satisfying certain conditions, with standard fiber $F$ a compact smooth manifold with boundary, constructing a stable transfer map $\Sigma^{\infty}B_{+} \to \Sigma^{\infty}E_{+}$. This is known as \emph{Becker-Gottlieb transfer.}
\end{itemize}

It is the hope of the author that the technology we have used in constructing the functor $\hat{r}$ will be useful in understanding similar constructions in other transfer theories.

\end{document}
