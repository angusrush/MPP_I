\documentclass[main.tex]{subfiles}

\begin{document}

\section{Application to presheaves}
\label{sec:application_to_presheaves}

\subsection{Introduction}
\label{ssc:introduction}

Consider the $\infty$-category $\S$ of spaces. By \cite[Lemma 6.1.1.1]{highertopostheory} and \cite[Lemma 2.4.7.12]{highertopostheory}, the functor
\begin{equation}
  \label{eq:ev1}
  \ev_{1}\colon \Map(\Delta^{1}, \S) \to \S
\end{equation}
is a bicartesian fibration. For each edge $f\colon \Delta^{1} \to \S$, corresponding to a morphism $f\colon X \to Y$ of spaces, the fiber of $\ev_{1}$ over $f$ corresponds to a pair of adjoint functor
\begin{equation*}
  f_{!} : \S^{/X} \longleftrightarrow \S^{/Y} : f^{*}
\end{equation*}
Interpreting $\S^{/X}$ and $\S^{/Y}$ as models for the $\P(X)$ and $\P(Y)$, the $\infty$-categories of presheaves on $X$ and $Y$, the functor $f^{*}$ corresponds to restriction, taking a presheaf $Y \to \S$ to the pullback $X \to Y \to \S$, and the functor $f_{!}$ takes a preasheaf $X \to \S$ to its left Kan extension along $X \to Y$.

In this section, we apply the construction of the previous chapter to build from these data a functor
\begin{equation}
  \label{eq:rhat}
  \hat{r}\colon \Span(\S) \to \ICat.
\end{equation}
This functor will take a space $X$ to the category $\P(X)$ of presheaves on $X$, and a span of spaces
\begin{equation*}
  \begin{tikzcd}
    & Y
    \arrow[dl, swap, "g"]
    \arrow[dr, "f"]
    \\
    X
    && X'
  \end{tikzcd}
\end{equation*}
to the map $\P(X) \to \P(Y)$ given by the composition
\begin{equation*}
  \begin{tikzcd}
    & \P(Y)
    \arrow[dr, "f^{*}"]
    \\
    \P(X)
    \arrow[ur, "g_{!}"]
    \arrow[rr, "f^{*} \circ g_{!}"]
    && \P(X').
  \end{tikzcd}.
\end{equation*}

Note that the category $\S$ has products. This induces a symmetric monoidal structure on $\S$, the Cartesian monoidal structure, which (as we will see) induces a symmetric monoidal structure on $\Span(\S)$. When we take $\Cat_{\infty}$ also to carry the Cartesian monoidal structure, we will see that our functor $\hat{r}$ (\hyperref[eq:rhat]{Equation~\ref*{eq:rhat}}) is even lax monoidal.

The idea behind the construction of the functor $\hat{r}$, in broad strokes, is as follows. We dress up the functor $\ev_{1}$, making it into a symmetric monoidal functor as follows (the monoidal structures on $\Fun(\Delta^{1}, \S)^{\otimes}$ and $\S^{\otimes}$ are not exactly the standard cocartesian monoidal functors, as we will later discuss; they are represented by \emph{cartesian,} rather than \emph{cocartesian} fibrations).
\begin{equation*}
  \begin{tikzcd}
    \Fun(\Delta^{1}, \S)^{\otimes}
    \arrow[rr, "r"]
    \arrow[dr, swap, "q"]
    && \S^{\otimes}
    \arrow[dl, "p"]
    \\
    & \Finp\op
  \end{tikzcd}
\end{equation*}

We then take spans in each of these categories, restricting the backwards- and forwards-facing legs judiciously so that we get a diagram
\begin{equation*}
  \begin{tikzcd}
    \Span(\Fun(\Delta^{1}, \S))^{\otimes}
    \arrow[rr, "\rho"]
    \arrow[dr, swap, "\varpi"]
    && \Span(\S)^{\otimes}
    \arrow[dl, "\pi"]
    \\
    & \Finp
  \end{tikzcd},
\end{equation*}
where the $\varpi$ and $\pi$ are symmetric monoidal structures and $\rho$ is a cocartesian fibration. In the language of \cite{luriehigheralgebra}, $\rho$ exhibits $\Span(\Fun(\Delta^{1}, \S))$ as a $\Span(\S)$-monoidal category. Restricting to the component of $\rho$ over $\langle 1 \rangle \in \Finp$, one finds a cocartesian fibration, which classifies a functor $\Span(\S) \to \ICat$. By \cite{luriehigheralgebra}, 2.4.2.4--2.4.2.6, this functor is lax monoidal, giving us our functor $\hat{r}$.

Much of the technology in this section is present in \cite{spectralmackeyfunctors2}. However, the manner in which we present it allows for generalization in a different direction than that carried out there. This will be the subject of future work.

\subsection{Defining the maps \texorpdfstring{$p$}{p}, \texorpdfstring{$q$}{q}, and \texorpdfstring{$r$}{r}}
\label{ssc:defining_the_maps_p_q_and_r}

Recall that for any $\infty$-category $\category{C}$ with finite products, one can construct a symmetric monoidal structure $\category{C}^{\times} \to \Finp$ whose corresponding tensor product $\times\colon \category{C} \times \category{C} \to \category{C}$ is the cartesian product; for more details, see \cite[Sec.\ 2.4.1]{luriehigheralgebra}. Similarly, as described in \cite[Sec.\ 2.4.2]{luriehigheralgebra}, for any $\infty$-category with coproducts there is a symmetric monoidal structure $\category{C}^{\amalg} \to \Finp$, called the \emph{cocartesian monoidal structure,} whose tensor product $\amalg\colon \category{C} \times \category{C} \to \category{C}$ is the coproduct. Because cocartesian fibrations give better control of the morphisms out of an object than morphisms into an object, the cocartesian monoidal structure is much more tractable than the cartesian monoidal structure.

Since taking coproducts in $\category{C}\op$ is the same as taking products in $\category{C}$, we can express the cartesian monoidal structure on $\category{C}$ as a cartesian fibration by taking the opposite of the cocartesian monoidal structure on $\category{C}\op$. In the case $\category{C} = \S$, this gives us the monoidal structure we are interested in.

\begin{definition}
  We define the map $p\colon \S_{\times} \to \Finp\op$ to be the functor
  \begin{equation*}
    ((\S\op)^{\amalg} \to \Finp)\op,
  \end{equation*}
  where $(\S\op)^{\amalg} \to \Finp$ is the cocartesian monoidal structure, as described in \cite[Sec.\ 2.4.2]{luriehigheralgebra}.
\end{definition}

In low degrees, the category $\S_{\times}$ admits the following description.
\begin{itemize}
  \item The objects are pairs $(\langle n \rangle, \vec{X})$, where $\langle n \rangle$ is an object of $\Finp$, and $\vec{X} = [X_{1}, \ldots, X_{n}]$ is an ordered $n$-tuple of objects of $\S$.

  \item The morphisms $(\langle m \rangle, \vec{X}) \to (\langle n \rangle, \vec{Y})$ are pairs $(\phi, \Psi)$, where $\phi\colon \langle n \rangle \to \langle m \rangle$ is a morphism in $\Finp$, and $\Psi$ consists, for each $i \in \langle m \rangle^{\circ}$, of a collection of morphisms $X_{i} \to Y_{j}$ for each $j \in \langle n \rangle^{\circ}$ with $\phi(j) = i$.
\end{itemize}

The functor $\S_{\times} \to \Finp$ is simply the obvious forgetful functor.

\begin{example}
  \label{eg:morphisms_in_cocartesian_monoidal_structure_on_spaces}
  An object $(\langle 1 \rangle, \vec{X})$ of $\S_{\times}$ in the fiber over $\langle 1 \rangle$ is the same thing as an object $X \in \langle S \rangle$. An object $( \langle 2 \rangle, \vec{Y})$ in the fiber over $\langle 2 \rangle$ is the same as a pair of objects $[Y, Y']$ in $\S$.

  A morphism $(\phi, \Psi)\colon (\langle 1 \rangle, \vec{X}) \to (\langle 2 \rangle, \vec{Y})$ where $\phi\colon \langle 2 \rangle \to \langle 1 \rangle$ is the morphism in $\Finp$ which sends $1 \mapsto 1$, $2 \mapsto 1$, consists of a pair of maps $X \to Y$, $X \to Y'$.
  \begin{equation*}
    \begin{tikzcd}[row sep=tiny]
      & 1
      \arrow[dl, mapsto]
      \\
      1
      \\
      & 2
      \arrow[ul, mapsto]
    \end{tikzcd}
    \qquad\qquad
    \begin{tikzcd}[row sep=tiny]
      & Y
      \\
      X
      \arrow[ur]
      \arrow[dr]
      \\
      & Y'
    \end{tikzcd}
  \end{equation*}
\end{example}

\begin{lemma}
  The map $p$ is a cartesian fibration, and a morphism $(\phi, \Psi)\colon (\langle m \rangle, \vec{X}) \to (\langle n \rangle, \vec{Y})$ in $\S_{\times}$ is $p$-cartesian if and only if it for each $i \in \langle m \rangle^{\circ}$, the maps $X_{i} \to Y_{j}$ exhibit $X_{i}$ as the product of the $Y_{j}$.
\end{lemma}
\begin{proof}
  This follows easily from the definition of the cocartesian monoidal structure.
\end{proof}

\begin{example}
  The morphism described in \hyperref[eg:morphisms_in_cocartesian_monoidal_structure_on_spaces]{Example~\ref*{eg:morphisms_in_cocartesian_monoidal_structure_on_spaces}} is cartesian if and only if $X \simeq Y \times Y'$, and the maps $X \to Y$ and $X \to Y'$ are the canonical projections.
\end{example}

\begin{definition}
  We can now define the maps $q$ and $r$ in one fell swoop via the following diagram, where the left-hand square is pullback. In particular, the space $\Fun(\Delta^{1}, \S)_{\times}$ is defined to be the (strict) pullback below.
  \begin{equation*}
    \begin{tikzcd}
      \Fun(\Delta^{1}, \S)_{\times}
      \arrow[r, hook]
      \arrow[rr, bend left, "r"]
      \arrow[d, swap, "q"]
      & \Fun(\Delta^{1}, \S_{\times})
      \arrow[r, "\ev_{1}"]
      \arrow[d]
      & \S_{\times}
      \arrow[d, "p"]
      \\
      \Finp\op
      \arrow[r, hook]
      \arrow[rr, bend right, swap, "\id"]
      & \Fun(\Delta^{1}, \Finp\op)
      \arrow[r, "\ev_{1}"]
      & \Finp\op
    \end{tikzcd}
  \end{equation*}
\end{definition}

\begin{lemma}
  The map $q$ is a Cartesian fibration, and the cartesian morphisms are those which are level-wise cartesian morphisms in $\S_{\times}$.
\end{lemma}
\begin{proof}
  The map $q$ is given by the pullback
  \begin{equation*}
    \begin{tikzcd}
      \Fun(\Delta^{1}, \category{S})_{\times}
      \arrow[r, hook]
      \arrow[d, swap, "q"]
      & \Fun(\Delta^{1}, \category{S}_{\times})
      \arrow[d, "q'"]
      \\
      \Finp\op
      \arrow[r, hook]
      & \Fun(\Delta^{1}, \Finp\op)
    \end{tikzcd}.
  \end{equation*}
  By \cite[Prop.\ 3.1.2.1]{highertopostheory}, the map $q'$ is a cartesian fibration whose cartesian morphisms are pointwise cartesian morphisms in $\S_{\times}$. The map $q$ is also a cartesian fibration, where a morphism in $\Fun(\Delta^{1}, \S)_{\times}$ is $q$-cartesian if and only if its image in $\Fun(\Delta^{1}, \S_{\times})$ is $q'$-cartesian. That every $q$ cartesian morphism is $q'$ cartesian follows from \cite[Prop\ 2.4.1.3]{highertopostheory}; to see other direction, consider a $q$-cartesian morphism in $\Fun(\Delta^{1}, \S)_{\times}$. 
\end{proof}

\begin{lemma}
  \label{lemma:map_on_cartesian_cats_is_bicart_fib}
  The map $r$ is a bicartesian fibration with $r$-cartesian morphisms pullback squares in $\S_{\times}$ and $r$-cocartesian morphisms squares of the form
  \begin{equation*}
    \begin{tikzcd}
      X
      \arrow[r, "\simeq"]
      \arrow[d]
      & Y
      \arrow[d]
      \\
      X'
      \arrow[r]
      & Y'
    \end{tikzcd}.
  \end{equation*}
\end{lemma}
\begin{proof}
  The map $r$ factors into
  \begin{equation*}
    \begin{tikzcd}
      \Fun(\Delta^{1}, \S)_{\times}
      \arrow[r, hook, "j"]
      & \Fun(\Delta^{1}, \S_{\times})
      \arrow[r, "\ev_{1}"]
      & \S_{\times}
    \end{tikzcd}
  \end{equation*}
  The map $\ev_{1}$ is a bicartesian fibration with the cartesian and cocartesian morphisms described above; the map $j$ is a full inclusion (in the sense of \hyperref[def:full_inclusion]{Definition~\ref*{def:full_inclusion}}), and therefore has the right-lifting property with respect to \emph{all} horn inclusions $\Lambda^{n}_{i} \hookrightarrow \Delta^{n}$, $n \geq 2$ (because $\Lambda^{n}_{i}$ contains all vertices for $n \geq 2$). Every edge in $\Fun(\Delta^{1}, \S)_{\times}$ is therefore both $j$-cartesian and $j$-cocartesian. \cite[Prop\ 2.4.1.3.3]{highertopostheory} implies that an edge in $\Fun(\Delta^{1}, \S)_{\times}$ is $r$-cartesian (resp.\ cocartesian) if and only if its image in $\Fun(\Delta^{1}, \S_{\times})$ is $\ev_{1}$-cartesian (resp.\ cocartesian).
\end{proof}

Let us take stock. We have now defined the data of the commutative diagram below. Here, $p$ and $q$ are cartesian fibrations, and $r$ is a bicartesian fibration.
\begin{equation*}
  \begin{tikzcd}
    \Fun(\Delta^{1}, \S)_{\times}
    \arrow[rr, "r"]
    \arrow[dr, swap, "q"]
    && \S_{\times}
    \arrow[dl, "p"]
    \\
    & \Finp\op
  \end{tikzcd}
\end{equation*}
Our next step is to take spans with judiciously-chosen legs in each of these quasicategories. To do this, we must define the subcategories to which the legs of our spans are allowed to belong. As in \hyperref[sec:segal_spaces_of_spans]{Section~\ref*{sec:segal_spaces_of_spans}}, we do this by defining triples of categories.

\subsection{Restricting the legs of the spans}
\label{ssc:restricting_the_legs_of_the_spans}

In this section, we define triple structures on the quasicategories $\Fun(\Delta^{1}, \S)_{\times}$, $\S_{\times}$, and $\Finp\op$.

We start by defining a triple structure on $\Finp\op$.

\begin{definition}
  We define a triple $\triple{F}$ as follows.
  \begin{itemize}
    \item $\mathcal{F} = \Finp\op$

    \item $\mathcal{F}\downdag = (\Finp\op)^{\simeq}$

    \item $\mathcal{F}\updag = \Finp\op$
  \end{itemize}
\end{definition}

Recall that in \hyperref[eg:spans_with_equivalences_on_one_leg]{Example~\ref*{eg:spans_with_equivalences_on_one_leg}}, we considered a triple structure on any quasicategory $\category{C}$ which produced spans in $\category{C}$ whose backwards-pointing legs were equivalences; this yielded a quasicategory $\Span^{\simeq}(\category{C})$ which was equivalent to $\category{C}$ via a weak categorical equivalence $\category{C} \to \Span^{\simeq}(\category{C})$, which we constructed explicitly. The triple $\triple{F}$ is similar to the triple structure used to define $\Span^{\simeq}(\category{C})$, except that the forward-facing legs are equivalences, rather than the backward-facing legs. A dual construction thus produces the following.

\begin{lemma}
  There is a weak Joyal equvialence
  \begin{equation*}
    \Finp \to \Span\triple{F},
  \end{equation*}
  which sends an $n$-simplex morphism $\langle m \rangle \to \langle n \rangle \to \cdots \to \langle p \rangle$ to a diagram
  \begin{equation*}
    \begin{tikzcd}
      & \langle n \ranglre
      \arrow[dr, equals]
      \\
      \langle m \rangle
      \arrow[ur, \phi]
      && \langle n \rangle
    \end{tikzcd},
  \end{equation*}
  where we have drawn the morphisms in the span as belonging to $\Finp$ rather than $\Finp\op$.
\end{lemma}

Next, we define a triple structure on $\S_{\times}$.

\begin{definition}
  We define a triple $\triple{P}$ as follows.
  \begin{itemize}
    \item $\mathcal{P} = \mathcal{S}_{\times}$

    \item $\mathcal{P}\downdag = \mathcal{S}_{\times} \times_{\Finp\op} (\Finp\op)^{\simeq}$

    \item $\mathcal{P}\updag = \mathcal{S}_{\times}$
  \end{itemize}
\end{definition}

Here:
\begin{itemize}
  \item $\mathcal{P} = \mathcal{S}_{\times}$, the Cartesian structure on the $\infty$-category of spaces presented as a Cartesian fibration, i.e.\ $((\S\op)^{\amalg})\op$

  \item $\mathcal{P}\downdag = \mathcal{S}_{\times} \times_{\Finp\op} (\Finp\op)^{\simeq}$

  \item $\mathcal{P}\updag = \mathcal{S}_{\times}$

  \item $\mathcal{M} = \Fun(\Delta^{1}, \S)_{\times}$, i.e.\ the pullback
    \begin{equation*}
      \begin{tikzcd}
        \Fun(\Delta^{1}, \mathcal{S})_{\times}
        \arrow[r]
        \arrow[d]
        & \Fun(\Delta^{1}, \mathcal{S}_{\times})
        \arrow[d]
        \\
        \Finp\op
        \arrow[r]
        & \Fun(\Delta^{1}, \Finp\op)
      \end{tikzcd}
    \end{equation*}

   \item $\mathcal{M}\downdag$ is given by the pullback
    \begin{equation*}
      \begin{tikzcd}
        \mathcal{M}\downdag
        \arrow[r]
        \arrow[d]
        & \mathcal{M}
        \arrow[d]
        \\
        (\Finp\op)^{\simeq}
        \arrow[r]
        & \Finp\op
      \end{tikzcd}.
    \end{equation*}

  \item $\mathcal{M}\updag$ is given by the subcategory of $\mathcal{M}$ on morphisms
    \begin{equation*}
      \begin{tikzcd}
        S
        \arrow[r, "f"]
        \arrow[d]
        & T
        \arrow[d]
        \\
        X
        \arrow[r]
        & Y
      \end{tikzcd}
    \end{equation*}
    such that $f$ is $p$-cartesian.\footnote{We need to make sure that over $\id_{\langle n \rangle}$, $f$ is an isomorphism, and this seems like the best way of doing this.}
\end{itemize}

The maps $p$, $q$, and $r$ are in particular maps $\mathcal{M} \to \mathcal{P} \to \mathcal{F}$ as follows:
\begin{equation*}
  \begin{tikzcd}
    \Fun(\Delta^{1}, \S)_{\times}
    \arrow[rr, "r"]
    \arrow[dr, swap, "q"]
    && \S_{\times}
    \arrow[dl, "p"]
    \\
    & \Finp\op
  \end{tikzcd}
\end{equation*}
It is not hard to check that $p$, $q$, and $r$ restrict to maps
\begin{equation*}
  \begin{tikzcd}[column sep=small]
    \mathcal{M}_{\dagger}
    \arrow[r, "r_{\dagger}"]
    \arrow[rr, swap,  bend right, "q_{\dagger}"]
    & \mathcal{P}_{\dagger}
    \arrow[r, "q_{\dagger}"]
    & \mathcal{F}_{\dagger}
  \end{tikzcd}
  \qquad\text{and}\qquad
  \begin{tikzcd}[column sep=small]
    \mathcal{M}^{\dagger}
    \arrow[r, "r^{\dagger}"]
    \arrow[rr, swap,  bend right, "q^{\dagger}"]
    & \mathcal{P}^{\dagger}
    \arrow[r, "q^{\dagger}"]
    & \mathcal{F}^{\dagger}
  \end{tikzcd}.
\end{equation*}



\begin{lemma}
  The morphism $p$ satisfies the conditions of the Omnibus theorem.
\end{lemma}
\begin{proof}
  We have to check the conditions of the Omnibus Theorem:
  \begin{enumerate}
    \item We have to show that for any isomorphism in $\Finp\op$, there exists a $p$-cocartesian lift. But since every morphism in $\mathcal{F}_{\dagger} = (\Finp\op)^{\simeq}$ is an equivalence, it suffices to find a $p$-cartesian lift, which is possible because $p$ is a Cartesian fibration.

    \item We need to show that for any commutative square
      \begin{equation*}
        \sigma \ = \quad
        \begin{tikzcd}
          S'
          \arrow[r, "f"]
          \arrow[d]
          & T'
          \arrow[d]
          \\
          S
          \arrow[r, "\simeq"]
          & T
        \end{tikzcd}
      \end{equation*}
      lying over a square
      \begin{equation*}
        \begin{tikzcd}
          \langle n \rangle
          & \langle n \rangle
          \arrow[l, swap, "\cong"]
          \\
          \langle m \rangle
          \arrow[u]
          & \langle m \rangle
          \arrow[l, swap, "\cong"]
          \arrow[u]
        \end{tikzcd}
      \end{equation*}
      in $\Finp\op$ (where we have drawn the arrows as they are in $\Finp$), $\sigma$ is a pullback square if and only if $f$ is $p$-cocartesian. But $f$ is lying over an equivalence, so it is $p$-cocartesian if and only if it is an equivalence, which in turn is true if and only if the square in question is pullback.
  \end{enumerate}
\end{proof}

\begin{lemma}
  The morphism $q$ satisfies the conditions of the Omnibus theorem.
\end{lemma}
\begin{proof}
  We have to check the conditions of the Omnibus Theorem:
  \begin{enumerate}
    \item We have to show that for any isomorphism in $\Finp\op$, there exists a $q$-cocartesian lift. But it suffices to find a $q$-cartesian lift, which is possible because $q$ is a Cartesian fibration.

    \item We need to show that for any commutative square in $\Fun(\Delta^{1}, \S)_{\times}$ corresponding to a cube in $\S_{\times}$ of the form
      \begin{equation*}
        \begin{tikzcd}
          X'
          \arrow[rr, "F"]
          \arrow[dd]
          \arrow[dr, "\phi"]
          && Y'
          \arrow[dd]
          \arrow[dr, "\psi"]
          \\
          & X
          \arrow[rr, crossing over, near start, "\simeq"]
          && Y
          \arrow[dd]
          \\
          S'
          \arrow[dr]
          \arrow[rr, near start, "f"]
          && T'
          \arrow[dr]
          \\
          & S
          \arrow[from=uu, crossing over]
          \arrow[rr, "\simeq"]
          && T
        \end{tikzcd}
      \end{equation*}
      (where $\phi$ is $p$-cartesian) and lying over a square in $\Finp\op$ of the form
      \begin{equation*}
        \begin{tikzcd}
          \langle n \rangle
          & \langle n \rangle
          \arrow[l, swap, "\cong"]
          \\
          \langle m \rangle
          \arrow[u]
          & \langle m \rangle
          \arrow[l, swap, "\cong"]
          \arrow[u]
        \end{tikzcd},
      \end{equation*}
      the morphism $(F, f)$ is $p$-cocartesian if and only if the top and bottom squares are pullback and $\psi$ is $p$-cocartesian. But this again follows because equivalences are $p$-cocartesian, and cocartesian morphisms are closed under composition.
  \end{enumerate}
\end{proof}

\begin{lemma}
  The map $r$ satisfies the conditions of the Omnibus Theorem.
\end{lemma}
\begin{proof}
  \leavevmode
  \begin{enumerate}
    \item It suffices to show that for each morphism $T \to S'$ in $\S_{\times}$ lying over $\langle m \rangle \overset{\cong}{\leftarrow} \langle m \rangle$ and each $Y \to T$ in the fiber over $\langle m \rangle$, the square
      \begin{equation*}
        \begin{tikzcd}
          Y
          \arrow[r, equals]
          \arrow[d]
          & Y
          \arrow[d]
          \\
          T
          \arrow[r]
          & S'
        \end{tikzcd}
      \end{equation*}
      is both $p$-cocartesian and $p_{\dagger}$-cocartesian, both of which follow from \hyperref[lemma:map_on_cartesian_cats_is_bicart_fib]{Lemma~\ref*{lemma:map_on_cartesian_cats_is_bicart_fib}}.

    \item We need to show that for any commutative square in $\Fun(\Delta^{1}, \S)_{\times}$ corresponding to a cube in $\S_{\times}$ of the form
      \begin{equation*}
        \begin{tikzcd}
          X'
          \arrow[rr, "F"]
          \arrow[dd]
          \arrow[dr, "\simeq"]
          && Y'
          \arrow[dd]
          \arrow[dr, "\psi"]
          \\
          & X
          \arrow[rr, crossing over, near start, "\simeq"]
          && Y
          \arrow[dd]
          \\
          S'
          \arrow[dr]
          \arrow[rr, near start, "f"]
          && T'
          \arrow[dr]
          \\
          & S
          \arrow[from=uu, crossing over]
          \arrow[rr]
          && T
        \end{tikzcd}
      \end{equation*}
      where the bottom square is pullback, the morphism $(F, f)$ is $r$-cocartesian if and only if the top square is pullback and $\psi$ is an equivalence. This is clear.
  \end{enumerate}
\end{proof}

This gives us the following commutative diagram.

\begin{equation*}
  \begin{tikzcd}
    \Span\triple{M}
    \arrow[rr, "\rho"]
    \arrow[dr, swap, "\varpi"]
    && \Span\triple{P}
    \arrow[dl, "\pi"]
    \\
    & \Span\triple{F}
  \end{tikzcd}
\end{equation*}

Recall that there is an equivalence $\Finp \to \Span(\triple{F})$. Pulling back along this map gives 

\end{document}
