\documentclass[main.tex]{subfiles}

\begin{document}

To do:
\begin{itemize}
  \item Fix the following bits of notation
    \begin{itemize}
      \item For $X$ a bisimplicial set, figure out what to do about $X_{n}$. Should it be called $X_{n, \bullet}$, or $X_{n}$, or $\Delta^{n} \backslash X$, or something else? $\ast$

      \item Marked simplicial sets: should a generic marked/doubly-marked simplicial set carry some notation (e.g.\ $X^{\dagger}$) to show that it carries some generic marking, rather than inventing some notation (like $\mathring{X}$) for underlying unmarked simplicial set?

      \item Should I notationally distinguish the marked and unmarked J+T constructions?
    \end{itemize}

  \item Figure out at which point each fibrancy condition becomes necessary.

  \item Does the existence of the Cartesian and Cocartesian marked model structure imply the existence of this one without messing around with bimarked simplicial sets?
\end{itemize}

\section{Cocartesian fibrations between complete Segal spaces}
\label{sec:cocartesian_fibrations_between_complete_segal_spaces}

\subsection{Marked bisimplicial sets}
\label{ssc:marked_bisimplicial_sets}

\begin{definition}
  A \defn{marked bisimplicial set} $(X, \mathcal{E})$ is a bisimplicial set $X$ together with a distinguished subset $\mathcal{E} \subseteq (X_{1})_{0}$ containing all degenerate edges, i.e.\ all edges in the image of $s_{0}\colon (X_{0})_{0} \to (X_{1})_{0}$. Equivalently, a marked bisimplicial set is bisimplicial set $X$ together with a marking $\mathcal{E}$ on the simplicial set given by the zeroth row of $X$, $X / \Delta^{0}$.
\end{definition}

\begin{definition}
  For a marked simplicial set $A$ and an unmarked simplicial set $B$, define a marking on the bisimplicial set $A \square B$ as follows: a simplex $(a, b) \in A_{1} \times B_{0}$ is marked if and only if $a$ is marked in $A$.
\end{definition}

This construction gives us a functor
\begin{equation*}
  - \square -  \colon \SSet^{+} \times \SSet \to \SSSet^{+}.
\end{equation*}

Our first order of business is to generalize the results of \cite{qcats_vs_segal_spaces} summarized in \hyperref[sss:the_box_functor]{Section~\ref*{sss:the_box_functor}} to the marked case. We will first show that the above functor is divisible on the left and on the right.

\begin{notation}
  For any marked simplicial set $A$, denote the underlying unmarked simplicial set by $\mathring{A}$. Similarly, for any marked bisimplicial set $X$, denote the underlying unmarked bisimplicial set by $\mathring{X}$.
\end{notation}

\begin{definition}
  Let $A$ denote a marked simplicial set, $B$ an unmarked simplicial set, and $X$ a marked bisimplicial set.
  \begin{itemize}
    \item Define an unmarked simplicial set $A \backslash X$ level-wise by
      \begin{equation*}
        (A \backslash X)_{n} = \Hom_{\SSSet^{+}}(A \square \Delta^{n}, X).
      \end{equation*}

    \item Define a marked simplicial set $X / B$ as follows. The underlying simplicial set is the same as $\mathring{X} / \mathring{B}$, and a 1-simplex $\Delta^{1} \to X / B$ is marked if and only if the corresponding map $\Delta^{1} \square B \to X$ of unmarked bisimplicial sets comes from a map of marked bisimplicial sets $(\Delta^{1})^{\sharp} \square B \to X$.
  \end{itemize}
\end{definition}

\begin{example}
  For any unmarked simplicial set $A$ and marked bisimplicial set $X$, there is an isomorphism
  \begin{equation*}
    A^{\flat} \backslash X \cong A \backslash \mathring{X}.
  \end{equation*}
  Similarly, for any unmarked bisimplicial set $Y$ and marked simplicial set $B$, there is an isomorphism
  \begin{equation*}
    B \backslash Y^{\sharp} \cong \mathring{B} \backslash Y.
  \end{equation*}
\end{example}

\begin{example}
  For any marked bisimplicial set $X$, the marked simplicial set $X / \Delta^{0}$ is the first row of $X$, together with the obvious marking. In particular, for any quasicategory $\category{C}$ with a marking $\mathcal{E}$, the bisimplicial set $\Gamma(\category{C})$ defined level-wise by
  \begin{equation*}
    \Gamma(\category{C})_{n} = \Map(\Delta^{n}, \category{C})^{\simeq}
  \end{equation*}
  is a complete Segal space, and the marking on $\category{C}$ gives a marking on $\Gamma(\category{C})$ in the obvious way. Then the first row $\Gamma(\category{C}) / \Delta^{0}$ is isomorphic $\category{C}$ as a marked simplicial set.
\end{example}

\begin{example}
  \label{eg:map_identity_on_first_component_is_map_of_marked}
  For any map $B \to B'$ of simplicial sets and any marked simplicial set $A$, the map $A \square B \to A \square B'$ is a map of marked simplicial sets.
\end{example}

\begin{proposition}
  We have the following adjunctions.

  \begin{enumerate}
    \item For each marked simplicial set $A \in \SSet^{+}$ there is an adjunction.
      \begin{equation*}
        A \square -\colon \SSet \leftrightarrow \SSSet^{+} : A \backslash -
      \end{equation*}

    \item For each unmarked simplicial set $B \in \SSet$ there is an adjunction.
      \begin{equation*}
        - \square B\colon \SSet^{+} \leftrightarrow \SSSet : - / B.
      \end{equation*}
  \end{enumerate}
\end{proposition}
\begin{proof}
  We prove each by exhibiting a bijection. In the first case, we have inclusions of subsets
  \begin{equation*}
    \Hom_{\SSSet^{+}}(A \square B, X) \subseteq \Hom_{\SSSet}(\mathring{A} \square B, \mathring{X}) \cong \Hom_{\SSet}(B, \mathring{A} \backslash \mathring{X}) \supseteq \Hom_{\SSet}(B, A \backslash X),
  \end{equation*}
  where the middle bijection is given by the maps $\Phi$ and $\Psi$ of \hyperref[prop:bijection_exhibiting_box_divisibility]{Proposition~\ref*{prop:bijection_exhibiting_box_divisibility}}. To show that there is a natural bijection between the subsets, it suffices to show that $\Phi$ and $\Psi$ restrict to maps between the subsets.

  To this end, suppose we have a map of bimarked simplicial sets $f\colon A \square B \to X$. We would like to show that under $\Phi$, this is taken to a map of simplicial sets $\tilde{f}\colon B \to A \backslash X$. The map $\Phi(f) = \tilde{f}$ takes an $n$-simplex $b\colon \Delta^{n} \to B$ to the composition
  \begin{equation*}
    \begin{tikzcd}
      A \square \Delta^{1}
      \arrow[r, "i"]
      & A \square B
      \arrow[r, "f"]
      & X
    \end{tikzcd}.
  \end{equation*}

  We need to check that this is really an $n$-simplex in $A \backslash X$, and not just $\mathring{A} \backslash \mathring{X}$, i.e.\ that it is a map of marked simplicial sets. That $i$ is a map of marked simplicial sets is clear by \hyperref[eg:map_identity_on_first_component_is_map_of_marked]{Example~\ref*{eg:map_identity_on_first_component_is_map_of_marked}}, and $f$ is a map of marked simplicial sets by assumption. Thus $\Phi(f)$ belongs to the subset we are interested in, and $\Phi$ restricts to a map between subsets.

  Now suppose we have a map $g\colon B \to A \backslash X$. Under $\Psi$, this is mapped to the composition
  \begin{equation*}
    \tilde{g}\colon
    \begin{tikzcd}
      A \square B
      \arrow[r, "\id \times f"]
      & A \square (A \backslash X)
      \arrow[r, "\ev"]
      & X.
    \end{tikzcd}
  \end{equation*}
  We need to check that this is a map of marked bisimplicial sets, i.e.\ that for each marked simplex $a \in A_{1}$ and each $b \in B_{0}$, the element $\tilde{g}(a, b)$ is marked in $X_{10}$. But $\tilde{g}(a, b) = g(b)_{10}(a, \id_{\Delta^{0}})$, which is marked because $g$ lands in $A \backslash X$ by assumption.

  Now we show the other bijection. Again we have subsets and a bijection
  \begin{equation*}
    \Hom_{\SSSet^{+}}(A \square B, X) \subseteq \Hom_{\SSSet}(\mathring{A} \square B, \mathring{X}) \cong \Hom_{\SSet}(\mathring{A}, B \backslash \mathring{X}) \supseteq \Hom_{\SSet^{+}}(A, B \backslash X),
  \end{equation*}
  where the bijection is given by the maps $\Phi'$ and $\Psi'$ from \hyperref[prop:bijection_exhibiting_box_divisibility]{Proposition~\ref*{prop:bijection_exhibiting_box_divisibility}}. As before, suppose that
  \begin{equation*}
    f\colon A \square B \to X
  \end{equation*}
  is a map of marked bisimplicial sets. Then $\Phi'(f) = \tilde{f}$ is defined by sending $\sigma \in A_{n}$ to the composition
  \begin{equation*}
    \begin{tikzcd}
      \Delta^{n} \square B
      \arrow[r, "{(\sigma, \id)}"]
      & A \square B
      \arrow[r, "f"]
      & X
    \end{tikzcd}.
  \end{equation*}
  We need to show that for each marked $a \in A_{1}$, the corresponding map
  \begin{equation*}
    \begin{tikzcd}
      \Delta^{1} \square B
      \arrow[r, "{(a, \id)}"]
      & A \square B
      \arrow[r, "f"]
      & X
    \end{tikzcd}
  \end{equation*}
  is a map of marked bisimplicial sets. The first map is because by assumption $a$ is marked, and the second is because we assumed that $f$ was marked.

  Now, let $g\colon A \to X / B$ be a map of marked simplicial sets. We need to check that the composition
  \begin{equation*}
    \begin{tikzcd}
      A \square B
      \arrow[r, "{(g, \id)}"]
      & (X / B) \square B
      \arrow[r, "\ev'"]
      & X
    \end{tikzcd}
  \end{equation*}
  takes marked edges to marked edges. Let $(a, b) \in A_{1} \times B_{0}$, with $a$ marked. This maps to
  \begin{equation*}
    (a, b) \mapsto (g(a), b) \mapsto g(a)_{10}(\id_{\Delta^{1}}, b) \in X_{10}.
  \end{equation*}
  By definition, $g(a)$ is a map of marked simplicial sets
  \begin{equation*}
    (\Delta^{1})^{\sharp} \square B \to X
  \end{equation*}
  which therefore sends $(\id_{\Delta^{1}}, b)$ to a marked edge in $X$ by assumption.
\end{proof}

This shows that the marked version of $\square$ is, in the language of \cite{qcats_vs_segal_spaces}, \emph{divisible on the left and on the right.} This implies that $A \backslash X$ is a functor of both $A$ and $X$, and that $X / B$ is a functor of both $X$ and $B$. This in turn implies that there is a bijection between maps
\begin{equation*}
  A \square B \to X,\qquad A \to X / B \qquad \text{and}\quad B \to A \backslash X.
\end{equation*}
Thus, the functors $- \backslash X$ and $X / -$ are mutually right adjoint. Since $\SSet$, $\SSet^{+}$, and $\SSSet^{+}$ are all finitely complete and cocomplete, the results of \cite[Sec.\ 7]{qcats_vs_segal_spaces} hold.

\begin{note}
  Toby: This would go on for a while. I'll probably reproduce the relevant parts of \cite{qcats_vs_segal_spaces} in a separate section, and then reference stuff there.
\end{note}

%Let $\category{C}$ be an $\infty$-category, and let $\Gamma(\category{C})$ be the complete Segal space
%\begin{equation*}
%  \Gamma(\category{C})_{n} \cong \Map(\Delta^{n}, \category{C})^{\simeq}.
%\end{equation*}
%Clearly, a marking on $\category{C}$ gives a marking on $\Gamma(\category{C})$.
%
%\begin{lemma}
%  Let $\category{C}$ be a quasicategory with a marking, and let $A$ be a marked simplicial set. There is an isomorphism of simplicial sets
%  \begin{equation*}
%    A \backslash \Gamma(\category{C}) \cong \Map^{+}(A, \category{C})^{\simeq},
%  \end{equation*}
%  where $\Map^{+}$ denotes the full subcategory of maps of marked simplicial sets. This isomorphism is natural in $A$. Furthermore, if the marking on $\category{C}$ contains all equivalences and is closed under composition, then the inclusion
%  \begin{equation*}
%    A \backslash \Gamma(\category{C}) \cong \Map^{+}(A, \category{C})^{\simeq} \subseteq \Map(A, \category{C})^{\simeq} \cong A^{\flat}  \backslash \Gamma(\category{C})
%  \end{equation*}
%  is a full subcategory inclusion of connected components.
%\end{lemma}

\begin{note}
  Toby: Is `full subcategory inclusion' the right terminology for the below? I mean that an $n$-simplex of $A^{\flat} \backslash X$ belongs to in $A \backslash X$ if and only if each of its vertices belong to $A^{\flat}$.
\end{note}

\begin{lemma}
  For any marked simplicial set $A$ and marked bisimplicial set $X$, the simplicial set $A \backslash X$ is a Kan complex, and the inclusion $i\colon A \backslash X \hookrightarrow A^{\flat} \backslash X \cong \mathring{A} \backslash \mathring{X}$ is a full subcategory inclusion.
\end{lemma}
\begin{proof}
  We first show that the map $i$ is a full subcategory inclusion. The $n$-simplices of $A \backslash X$ are maps of marked simplicial sets $\tilde{\sigma}\colon A \square \Delta^{n} \to X$. A map of underlying bisimplicial sets gives a map of marked bisimplicial sets if and only if it respects the markings, i.e.\ if and only if for each $(a, i) \in A_{1} \times (\Delta^{n})_{0}$ with $a$ marked, $\tilde{\sigma}(a, i)$ is marked in $X$. This is equivalent to demanding that $\sigma|_{\Delta^{\{i\}}}$ belong to $A \backslash X$.

  To show that $A \backslash X$ is a Kan complex, we need to find dashed lifts below.
  \begin{equation*}
    \begin{tikzcd}
      \Lambda^{n}_{k}
      \arrow[r]
      \arrow[d, hook]
      & A \backslash X
      \\
      \Delta^{n}
      \arrow[ur, dashed]
    \end{tikzcd}.
  \end{equation*}
  For $n = 1$, the horn inclusion is of the form $\Delta^{0} \hookrightarrow \Delta^{1}$, and we can take the lift to be degenerate. For $n \geq 2$, we can augment our diagram as follows.
  \begin{equation*}
    \begin{tikzcd}
      \Lambda^{n}_{k}
      \arrow[r]
      \arrow[d, hook]
      & A \backslash X
      \arrow[r]
      & A^{\flat} \backslash X
      \\
      \Delta^{n}
      \arrow[urr, dashed]
    \end{tikzcd}.
  \end{equation*}
  Since $A^{\flat} \backslash X$ is a Kan complex, we can always find such a dashed lift. The inclusion $\Lambda^{n}_{k} \hookrightarrow \Delta^{n}$ is full on vertices, so our lift factors through $A \backslash X$.
\end{proof}

\begin{definition}
  Let $X$ be a bisimplicial set, and let $\mathcal{E}$ be a marking on $X$. We will say that $\mathcal{E}$ \defn{respects path components} if it has the following property: for any map $\Delta^{1} \to X_{1}$ representing an edge $e \to e'$ between morphisms $e$ and $e'$, the morphism $e$ is marked if and only if the morphism $e'$ is marked.
\end{definition}

\begin{proposition}
  \label{prop:cartesian_marking_respects_path_components}
  Let $f\colon X \to Y$ be a Reedy fibration between marked bisimplicial sets such that the marking on $X$ respects path components, and let $u\colon A \to A'$ be a morphism of marked simplicial sets whose underlying morphism of unmarked simplicial sets is a monomorphism. Then the map $\langle u \backslash f \rangle$ is a Kan fibration.
\end{proposition}
\begin{proof}
  We need to show that for each $n \geq 0$ and $0 \leq k \leq n$ we can solve the lifting problem
  \begin{equation*}
    \begin{tikzcd}
      \Lambda^{n}_{k}
      \arrow[r]
      \arrow[d]
      & A' \backslash X
      \arrow[d]
      \\
      \Delta^{n}
      \arrow[r]
      \arrow[ur, dashed]
      & A \backslash X \times_{A' \backslash Y} A' \backslash Y
    \end{tikzcd}.
  \end{equation*}
  First assume that $n \geq 2$. We can augment the above square as follows.
  \begin{equation*}
    \begin{tikzcd}
      \Lambda^{n}_{k}
      \arrow[r]
      \arrow[d]
      & A' \backslash X
      \arrow[r]
      \arrow[d]
      & A^{\flat} \backslash X
      \arrow[d]
      \\
      \Delta^{n}
      \arrow[r]
      & A \backslash X \times_{A' \backslash Y} A' \backslash Y
      \arrow[r]
      & A^{\flat} \backslash X \times_{(A')^{\flat} \backslash Y} (A')^{\flat} \backslash X
    \end{tikzcd}.
  \end{equation*}
  Since the map on the right is a Kan fibration, we can solve the outer lifting problem. All the vertices of $\Delta^{n}$ belong to $\Lambda^{n}_{k}$, so a lift of the outside square factors through $A' \backslash X$.

  Now take $n = 1$, $k = 0$, so our horn inclusion is $\Delta^{\{0\}} \hookrightarrow \Delta^{1}$. We need to be able to solve the lifting problem
  \begin{equation*}
    \begin{tikzcd}
      A
      \arrow[r]
      \arrow[d]
      & X / \Delta^{1}
      \arrow[d]
      \\
      A'
      \arrow[r]
      \arrow[ur, dashed]
      & X / \Delta^{0} \times_{Y / \Delta^{0}} Y / \Delta^{1}
    \end{tikzcd}.
  \end{equation*}
  Note that a lift always exists on the level of simplicial sets, so we only need to check that any such lift respects the marking on $X$. To see this, consider the following triangle formed by some dashed lift.
  \begin{equation*}
    \begin{tikzcd}[row sep=small, column sep=large]
      & X / \Delta^{1}
      \arrow[dd]
      \\
      A'
      \arrow[ur, dashed]
      \arrow[dr]
      \\
      & X / \Delta^{0}
    \end{tikzcd}
  \end{equation*}
  Let $a \in A_{1}$ be a marked 1-simplex, and consider the diagram
  \begin{equation*}
    \begin{tikzcd}[row sep=small, column sep=large]
      \Delta^{1} \square \Delta^{0}
      \arrow[r, "{(a, \id)}"]
      \arrow[dd]
      & A \square \Delta^{0}
      \arrow[dd]
      \arrow[dr]
      \\
      && X
      \\
      \Delta^{1} \square \Delta^{1}
      \arrow[r, "{(a, \id)}"]
      & A \square \Delta^{1}
      \arrow[ur, dashed, swap, "\ell"]
    \end{tikzcd},
  \end{equation*}
  where the triangle on the right is the adjunct to the triangle above. In order to check that the dashed lift respects the marking on $X$, we have to show that for each $(a, b) \in (A \square \Delta^{1})_{10} = A_{1} \times \{0, 1\}$, the morphism $\ell(a, b)$ is marked in $X$. The commutativity of the triangle guarantees this for $b = 0$. The map $\Delta^{1} \square \Delta^{1} \to X$ gives us a 1-simplex in $X_{1}$ connecting $\ell(a, 0)$ and $\ell(a, 1)$, which implies by that $\ell(a, 1)$ is also marked because each marking respects path components.

  The case $n = 1$, $k = 1$ is essentially identical.
\end{proof}

\subsection{Cocartesian morphisms}
\label{ssc:cocartesian_morphisms}

\begin{definition}
  Let $f\colon X \to Y$ be a Reedy fibration between complete Segal spaces. A morphism $e \in X_{10}$ is \defn{$f$-cocartesian} if the square
  \begin{equation*}
    \begin{tikzcd}
      \Delta^{2} \backslash X \times_{\Delta^{\{0, 1\}} \backslash X} \{e\}
      \arrow[r]
      \arrow[d]
      & \Lambda^{2}_{0} \backslash X \times_{\Delta^{\{0, 1\}} \backslash X} \{e\}
      \arrow[d]
      \\
      \Delta^{2} \backslash Y \times_{\Delta^{\{0, 1\}} \backslash Y} \{fe\}
      \arrow[r]
      & \Lambda^{2}_{0} \backslash Y \times_{\Delta^{\{0, 1\}} \backslash Y} \{fe\}
    \end{tikzcd}
  \end{equation*}
  is homotopy pullback.
\end{definition}

\begin{example}
  Identity morphisms are clearly $f$-cocartesian.
\end{example}

\begin{definition}
  For $n \geq 1$, define the following simplicial subsets of $\Delta^{n}$.

  \begin{itemize}
    \item For $n \geq 1$, denote by $I_{n}$ the \defn{spine} of $\Delta^{n}$, i.e.\ the simplicial subset
      \begin{equation*}
        \Delta^{\{0, 1\}} \amalg_{\Delta^{\{1\}}} \Delta^{\{1, 2\}} \amalg_{\Delta^{\{2\}}} \cdots \amalg_{\Delta^{\{n-1\}}} \Delta^{\{n-1, n\}} \subseteq \Delta^{n}.
      \end{equation*}

    \item For $n \geq 2$, denote by $L_{n}$ the simplicial subset
      \begin{equation*}
        L_{n} = \Delta^{\{0, 1\}} \amalg_{\Delta^{\{0\}}} \Delta^{\{0, 2\}} \amalg_{\Delta^{\{2\}}} \overbrace{\Delta^{\{2, 3\}} \amalg_{\Delta^{\{3\}}}\cdots \amalg_{\Delta^{\{n-1\}}} \Delta^{\{n-1, n\}}}^{I_{\{2, \ldots, n\}}} \subseteq \Delta^{n}.
      \end{equation*}
      That is, $L_{n}$ is the union of $\Delta^{\{0, 1\}}$ with the spine of $d_{1}\Delta^{n}$. We will call $L_{n}$ the \defn{left spine} of $\Delta^{n}$.
  \end{itemize}
\end{definition}

Note that $L_{2} \cong \Lambda^{2}_{0}$.

%\begin{lemma}
%  For any $n \geq 2$, the map
%  \begin{equation*}
%    \Delta^{\{0, \ldots, n-1\}} \amalg_{\Delta^{\{n-1\}}} \Delta^{\{n-1, n\}} \hookrightarrow \Delta^{n}
%  \end{equation*}
%  is inner anodyne.
%\end{lemma}
%\begin{proof}
%  This map can be written as the starred smash product of $\emptyset \hookrightarrow \Delta^{n-2}$, which is an inclusion, and $\Delta^{\{0\}} \hookrightarrow \Delta^{1}$, which is left anodyne. The result follows from \cite[Lemma 2.1.2.3]{highertopostheory}.
%\end{proof}

\begin{note}
  Toby: I don't know if the below proof is easy enough that it doesn't need to be expanded. I started typing out a more detailed proof, but the the notation became pretty terrible. This should probably be taken as another sign that better notation is needed.
\end{note}

\begin{proposition}
  \label{prop:only_lowest_lifting_condition_is_necessary}
  Let $f\colon X \to Y$ be a Reedy fibration between Segal spaces, and let $e \in X_{10}$ be an $f$-cocartesian morphism. Then the square
  \begin{equation*}
    \begin{tikzcd}
      \Delta^{n} \backslash X \times_{\Delta^{\{0, 1\}} \backslash X} \{e\}
      \arrow[r]
      \arrow[d]
      & L_{n} \backslash X \times_{\Delta^{\{0, 1\}} \backslash X} \{e\}
      \arrow[d]
      \\
      \Delta^{n} \backslash Y \times_{\Delta^{\{0, 1\}} \backslash Y} \{fe\}
      \arrow[r]
      & L_{n} \backslash Y \times_{\Delta^{\{0, 1\}} \backslash Y} \{fe\}
    \end{tikzcd}
  \end{equation*}
  is homotopy pullback for all $n \geq 2$.
\end{proposition}
\begin{proof}
  We have the base case $n = 2$ because $e$ is $f$-cocartesian. The higher cases follow from the Segal condition.

  % Assume the result is true for $n - 1$. Then the square
  % \begin{equation*}
  %   \begin{tikzcd}
  %     (\Delta^{n-1} \backslash X \times_{\Delta^{\{0, 1\}} \backslash X} \{e\}) \times_{\Delta^{\{n-1\}} \backslash X} \Delta^{\{n-1, n\}} \backslash X
  %     \arrow[r]
  %     \arrow[d]
  %     & L_{n} \backslash X \times_{\Delta^{\{0, 1\}} \backslash X} \{e\}
  %     \arrow[d]
  %     \\
  %     \Delta^{n-1} \backslash Y \times_{\Delta^{\{0, 1\}} \backslash Y} \{fe\}
  %     \arrow[r]
  %     & L_{n} \backslash Y \times_{\Delta^{\{0, 1\}} \backslash Y} \{fe\}
  %   \end{tikzcd}
  % \end{equation*}
\end{proof}

For any simplicial subset $A \subseteq \Delta^{n}$, denote the marking on $A$ where the only marked nondegenerate edge is $\Delta^{\{0, 1\}}$ by $A^{\mathcal{L}}$. If $A$ does not contain the edge $\Delta^{\{0, 1\}}$ then this agrees with the $\flat$-marking. For any simplicial set $A$, define a marked simplicial set $(\Delta^{1} \star A, \mathcal{L'})$ where the only nondegenerate simplex belonging to $\mathcal{L'}$ is $\Delta^{1}$.

\begin{lemma}
  \label{lemma:starred_smash_with_mono}
  Let $A \hookrightarrow B$ be a monomorphism of simplicial sets, and suppose that $B$ is $n$-skeletal (and therefore that $A$ is $n$-skeletal). Then the map
  \begin{equation*}
    \begin{tikzcd}
      (\Delta^{\{0\}} \star B)^{\flat} \coprod_{(\Delta^{\{0\}} \star A)^{\flat}} (\Delta^{1} \star A)^{\mathcal{L}'} \hookrightarrow (\Delta^{1} \star B)^{\mathcal{L}'}
    \end{tikzcd}
  \end{equation*}
  is in the saturated hull of the morphisms
  \begin{equation*}
    (\Lambda^{k}_{0})^{\mathcal{L}} \hookrightarrow (\Delta^{k})^{\mathcal{L}},\qquad 2 \leq k \leq n+2.
  \end{equation*}
\end{lemma}
\begin{proof}
  It suffices to show this for $A \hookrightarrow B = \partial \Delta^{m} \hookrightarrow \Delta^{m}$ for $0 \leq m \leq n$. In this case the necessary map is of the form
  \begin{equation*}
    (\Lambda^{m+2}_{0})^{\mathcal{L}} \hookrightarrow (\Delta^{m+2})^{\mathcal{L}}.
  \end{equation*}
\end{proof}

We will say a collection of morphisms $\mathcal{A} \subset \mathrm{Mor}(\SSet^{+})$ has the \emph{right cancellation property} if for all $u$, $v \in \mathrm{Mor}(\SSet^{+})$,
\begin{equation*}
  u \in \mathcal{A},\quad vu \in \mathcal{A} \quad \implies \quad v \in A.
\end{equation*}

\begin{lemma}
  \label{lemma:saturated_hull_of_left_spine_inclusions}
  Let $\mathcal{A}$ be a saturated set of morphisms of $\SSet^{+}$ all of whose underlying morphisms are monomorphisms, and which has the right cancellation property. Further suppose that $\mathcal{A}$ contains the following classes of morphisms.
  \begin{enumerate}
    \item Maps $(A)^{\flat} \hookrightarrow (B)^{\flat}$, where $A \to B$ is inner anodyne.

    \item Left spine inclusions $(L_{n})^{\mathcal{L}} \hookrightarrow (\Delta^{n})^{\mathcal{L}}$, $n \geq 2$.
  \end{enumerate}

  Then $\mathcal{A}$ contains left horn inclusions $(\Lambda^{n}_{0})^{\mathcal{L}} \hookrightarrow (\Delta^{n})^{\mathcal{L}}$, $n \geq 2$.
\end{lemma}
\begin{proof}
  We have an isomorphism $(L_{2})^{\mathcal{L}} \cong (\Lambda^{2}_{1})^{\mathcal{L}}$, which belongs to $\mathcal{A}$ because saturated sets contain all isomorphisms.

  We proceed by induction. Suppose we have shown that all horn inclusions $(\Lambda^{k}_{0})^{\mathcal{L}} \hookrightarrow (\Delta^{k})^{\mathcal{L}}$ belong to $\mathcal{A}$ for $2 \leq k < n$. From now on on we will suppress the marking $(-)^{\mathcal{L}}$. All simplicial subsets of $\Delta^{n}$ below will have $\Delta^{\{0, 1\}}$ marked if they contain it.

  Consider the factorization
  \begin{equation*}
    \begin{tikzcd}
      L_{n}
      \arrow[r, "u_{n}"]
      \arrow[rr, bend right, swap, "v_{n} \circ u_{n}"]
      & \Lambda^{n}_{0}
      \arrow[r, "v_{n}"]
      & \Delta^{n}
    \end{tikzcd}.
  \end{equation*}
  The morphism $v_{n} \circ u_{n}$ belongs to $\mathcal{A}$ by assumption, so in order to show that $v_{n}$ belongs to $\mathcal{A}$, it suffices by right cancellation to show that $u_{n}$ belongs to $\mathcal{A}$. Consider the factorization
  \begin{equation*}
    \begin{tikzcd}
      L_{n}
      \arrow[r, "w'_{n}"]
      & L_{n} \cup d_{1} \Delta^{n}
      \arrow[r, "w_{n}"]
      & \Lambda^{n}_{0}.
    \end{tikzcd}
  \end{equation*}
  The map $w'_{n}$ is a pushout along a spine inclusion, and hence is inner anodyne. Hence, we need only show that $w_{n}$ belongs to $\mathcal{A}$. Let
  \begin{equation*}
    Q = d_{2} \Delta^{n} \cup \cdots \cup d_{n} \Delta^{n},
  \end{equation*}
  and consider the following pushout diagram.
  \begin{equation*}
    \begin{tikzcd}
      (L_{n} \cup d_{1}\Delta^{n}) \cap Q
      \arrow[r, hook]
      \arrow[d, hook]
      & Q
      \arrow[d, hook]
      \\
      L_{n} \cup d_{1} \Delta^{n}
      \arrow[r, hook]
      & L_{n} \cup d_{1}\Delta^{n} \cup Q
    \end{tikzcd}.
  \end{equation*}
  Since $L_{n} \cup d_{1} \Delta^{n} \cup Q \cong \Lambda^{n}_{0}$, the bottom map is $w_{n}$, so it suffices to show that the top map belongs to $\mathcal{A}$. But this is isomorphic to
  \begin{equation*}
    \begin{tikzcd}
      (\Delta^{\{0, 1\}} \star \emptyset) \coprod_{(\Delta^{\{0\}} \star \emptyset)} (\Delta^{\{0\}} \star \partial \Delta^{\{2, 3, \ldots, n\}}) \hookrightarrow \Delta^{\{0, 1\}} \star \partial \Delta^{\{2, 3, \ldots, n\}}.
    \end{tikzcd}
  \end{equation*}
  The simplicial set $\partial \Delta^{\{2, \ldots, n\}}$ is $(n-3)$-skeletal, so this map belongs to $\mathcal{A}$ by \hyperref[lemma:starred_smash_with_mono]{Lemma~\ref*{lemma:starred_smash_with_mono}}.
\end{proof}

\begin{definition}
  \label{def:cocartesian_fibration_between_complete_segal_spaces}
  Let $f\colon X \to Y$ be a Reedy fibration between complete Segal spaces. We will say that $f$ is a \defn{cocartesian fibration} if each morphism in $Y$ has an $f$-cocartesian lift in $X$. More explicitly, $f$ is a cocartesian fibration if it is a Reedy fibration with the following property: for each edge $e\colon y \to y'$ in $Y$ and each vertex $x \in X$ such that $f(x) = y$, there exists an $f$-cocartesian morphism $\tilde{e}\colon x \to x'$ such that $f(\tilde{e}) = e$.
\end{definition}

\begin{definition}
  \label{def:cocoartesian_marking_on_complete_segal_spaces}
  Let $f\colon X \to Y$ be a cocartesian fibration between complete Segal spaces. We define a marking on $X$ such that a morphism is marked if and only if it is $f$-cocartesian. We will denote the corresponding marked bisimplicial by set $X^{\natural}$. We can view the map $f$ as a map of marked bisimplicial sets $f\colon X^{\natural} \to Y^{\sharp}$.
\end{definition}

\begin{note}
  Toby: In the rest of this subsection, we have to assume that the cocartesian marking respects path components. I'm convinced that this isn't necessary (i.e.\ that the cocartesian marking always respects path components), but this doesn't matter in our case as it is trivially true for spans.
\end{note}

\begin{lemma}
  Let $f\colon X \to Y$ be a Reedy fibration between bisimplicial sets, and let $e \in X_{10}$ be an edge. Then
  \begin{equation*}
    A 
  \end{equation*}
\end{lemma}

\begin{proposition}
  \label{prop:segal_cocartesian_morphisms_are_quasicategory_cocartesian}
  Let $f\colon X \to Y$ be a cocartesian fibration of complete Segal spaces, and further assume that the cocartesian marking on $X$ respects path components. This is equivalent to demanding that if a morphism in $X$ is $f$-cocartesian, then each morphism in its the path component is $f$-cocartesian. Let $e \in X_{10}$ be an $f$-cocartesian morphism. Then $e$, viewed as a morphism in the quasicategory $X / \Delta^{0}$, is $f / \Delta^{0}$-cocartesian.
\end{proposition}
\begin{proof}
  Let $u^{n}_{0}\colon (\Lambda^{n}_{0} \hookrightarrow \Delta^{n})^{\mathcal{L}}$ denote the $\mathcal{L}$-marked inclusion. First, let $n \geq 2$, and consider the set
  \begin{equation*}
    S =
    \left\{
      \substack{
        u\colon A \to \text{ B morphism of} \\
        \text{marked simplicial sets} \\
        \text{such that $\mathring{u}$ is mono}
      }
      \ \bigg| \
      \langle u \backslash f \rangle \text{ weak homotopy equivalence}
    \right\}.
  \end{equation*}
  It is clear that this set has the right cancellation property. By \hyperref[prop:cartesian_marking_respects_path_components]{Proposition~\ref*{prop:cartesian_marking_respects_path_components}}, a map $u$ belonging to $S$ is automatically a Kan fibration, hence is a trivial Kan fibration. Thus, we can equivalently say that $u \in S$ if and only if $u$ has the left-lifting property with respect to all maps of the form $\langle X / v \rangle$, where $v$ is a cofibration of simplicial sets. Thus, $S$ is saturated.

  The set $S$ contains all flat-marked inner anodyne morphisms because $f$ is a Reedy fibration. \hyperref[prop:only_lowest_lifting_condition_is_necessary]{Proposition~\ref*{prop:only_lowest_lifting_condition_is_necessary}} implies that $S$ contains all left spine inclusions $(L_{n})^{\mathcal{L}} \hookrightarrow (\Delta^{n})^{\mathcal{L}}$, $n \geq 2$. Thus, by \hyperref[lemma:saturated_hull_of_left_spine_inclusions]{Lemma~\ref*{lemma:saturated_hull_of_left_spine_inclusions}}, $S$ contains all $\mathcal{L}$-marked left horn inclusions.

\end{proof}

\begin{corollary}
  \label{cor:cocart_fib_between_css_gives_cocart_fib_of_quasicats}
  Let $f\colon X \to Y$ be a cocartesian fibration of complete Segal spaces such that the cocartesian marking on $X$ respects path components. Then the map
  \begin{equation*}
    f/\Delta^{0}\colon X/\Delta^{0} \to Y/\Delta^{0}
  \end{equation*}
  is a cocartesian fibration of quasicategories.
\end{corollary}
\begin{proof}
  By \hyperref[thm:inner_fibration_between_quasicategories]{Theorem~\ref*{thm:inner_fibration_between_quasicategories}}, the map $f/\Delta^{0}$ is an inner fibration between quasicategories. By assumption, every morphism in $Y$ has a $f$-cartesian lift. By \hyperref[prop:segal_cocartesian_morphisms_are_quasicategory_cocartesian]{Proposition~\ref*{prop:segal_cocartesian_morphisms_are_quasicategory_cocartesian}}, these lifts are also $f / \Delta^{0}$-cocartesian.
\end{proof}

\section{Complete Segal spaces of spans}
\label{sec:complete_segal_spaces_of_spans}

The main goal of this section is to prove the following theorem.

\begin{theorem}
  \label{thm:span_of_bicartesian_fibration_is_bicartesian_fibration}
  Let $\category{C}$ and $\category{D}$ be quasicategories admitting all pullbacks, and let $p\colon \category{C} \to \category{D}$ be a bicartesian fibration which preserves pullbacks. Further suppose that $p$ has the following property:

  \begin{quote}
    For any square
    \begin{equation*}
      \sigma =
      \begin{tikzcd}
        x
        \arrow[r, "f"]
        \arrow[d]
        & y
        \arrow[d]
        \\
        x'
        \arrow[r, "f'", "\bullet" marking]
        & y'
      \end{tikzcd}
    \end{equation*}
    in $\category{C}$ such that the morphism $f'$ is $p$-cocartesian and $p(\sigma)$ is pullback in $\category{D}$, the following are equivalent.
    \begin{itemize}
      \item The morphism $f$ is $p$-cocartesian.

      \item The square $\sigma$ is pullback.
    \end{itemize}
  \end{quote}

  Then the functor $\pi\colon \Span(\category{C}) \to \Span(\category{D})$ is a cocartesian fibration of Segal spaces, and if a morphism has the form
  \begin{equation*}
    \begin{tikzcd}
      & y
      \arrow[dl, "\circ" marking]
      \arrow[dr, "\bullet" marking]
      \\
      x
      && x'
    \end{tikzcd},
  \end{equation*}
  then it is $p$-cocartesian.
\end{theorem}

\begin{note}
  Since the definition of $\Span(\category{C})$ is self-dual, the functor $\Span(\category{C}) \to \Span(\category{D})$ is also a cartesian fibration with cartesian morphisms of the form
  \begin{equation*}
    \begin{tikzcd}
      & y
      \arrow[dl, "\bullet" marking]
      \arrow[dr, "\circ" marking]
      \\
      x
      && x'
    \end{tikzcd}.
  \end{equation*}
\end{note}

We prove \hyperref[thm:span_of_bicartesian_fibration_is_bicartesian_fibration]{Theorem~\ref*{thm:span_of_bicartesian_fibration_is_bicartesian_fibration}} in several steps. Since we will be working with bicartesian fibrations, it will be helpful to adapt some of the tools of marked simplicial sets to our purposes.

\subsection{Doubly-marked simplicial sets}
\label{ssc:doubly-marked_simplicial_sets}

Our proof of \hyperref[thm:span_of_bicartesian_fibration_is_bicartesian_fibration]{Theorem~\ref*{thm:span_of_bicartesian_fibration_is_bicartesian_fibration}} will involve working with with bicartesian fibrations. For this reason, it will be helpful to have results about simplicial sets with two markings, one of which controls the cocartesian structure and one of which controls the cartesian structure. We will call such simplicial sets \emph{doubly-marked;} the traditional terminology for this is \emph{bimarked,} but we wish to avoid the potential confusion between the similar terms \emph{marked bisimplicial set} and \emph{bimarked simplicial set.} This section consists mainly of verifications that some key results about marked simplicial sets which can be found in \cite[Sec.\ 3.1]{highertopostheory} hold in the doubly-marked case. The only results we will make use of are \hyperref[proposition:doubly-marked_anodyne_homotopy_pullback]{Proposition~\ref*{proposition:doubly-marked_anodyne_homotopy_pullback}}, \hyperref[eg:bimarking_on_asd_delta1]{Example~\ref*{eg:bimarking_on_asd_delta1}}, and \hyperref[eg:bimarking_on_asd_delta2]{Example~\ref*{eg:bimarking_on_asd_delta2}}.

\begin{definition}
  A \defn{doubly-marked simplicial set} is a triple $(X, \mathcal{E}, \mathcal{E}')$, where $X$ is a simplicial set and $\mathcal{E}$ and $\mathcal{E}'$ are markings. We will often shorten this to $X^{(\mathcal{E}, \mathcal{E}')}$. A morphism of doubly-marked simplicial sets is a morphism of the underlying simplicial sets which preserves each class of markings separately. We will denote the category of doubly-marked simplicial sets by $\SSet^{++}$.
\end{definition}

\begin{example}
  For any simplicial set $X$ we will denote
  \begin{itemize}
    \item The doubly-marked simplicial set where $\mathcal{E}$ and $\mathcal{E}'$ contain only the degenerate edges by $X^{( \flat, \flat )}$,

    \item The doubly-marked simplicial set where $\mathcal{E}$ contains only the degenerate edges and $\mathcal{E}'$ contains every edge by $X^{( \flat, \sharp )}$,

    \item The doubly-marked simplicial set where $\mathcal{E}$ contains every edge and $\mathcal{E}'$ contains only the degenerate edges by $X^{( \sharp, \flat )}$, and

    \item The doubly-marked simplicial set where $\mathcal{E}$ and $\mathcal{E}'$ contain every edge by $X^{( \sharp, \sharp )}$.
  \end{itemize}
\end{example}

\begin{example}
  \label{eg:bicartesian_marking}
  Let $p\colon \category{C} \to \category{D}$ be a bicartesian fibration between quasicategories. Denote by $\category{C}^{\natural}$ the doubly-marked simplicial set where
  \begin{itemize}
    \item The set $\mathcal{E}$ is the set of all $p$-cocartesian morphisms, and

    \item The set $\mathcal{E}'$ is the set of all $p$-cartesian morphisms.
  \end{itemize}

  In this way every bicartesian fibration gives a morphism of doubly-marked simplicial sets.
\end{example}

%The category $\SSet^{++}$ is closely connected to the category $\SSet^{+}$ and the category $\SSet$. We have the following obvious results.
%\begin{itemize}
%  \item There is a forgetful functor $u\colon \SSet^{++} \to \SSet$ which forgets both markings. This has left adjoint $\iota\colon X \mapsto (X, \flat, \flat)$. The functor $\iota$ is a full subcategory inclusion.
%
%  \item There is a forgetful functor $u_{1}\colon \SSet^{++} \to \SSet^{+}$, which forgets the second marking, sending $(X, \mathcal{E}, \mathcal{E}') \mapsto (X, \mathcal{E})$. This has a left adjoint $\iota_{1}$ given by the functor which sends $(X, \mathcal{E}) \mapsto (X, \mathcal{E}, \flat)$. The functor $\iota_{1}$ is a full subcategory inclusion; there is a bijection between maps between marked simplicial sets $(X, \mathcal{E}) \to (Y, \mathcal{E}')$ and maps $(X, \mathcal{E}, \flat) \to (Y, \mathcal{E}', \flat)$. The same is true of the functor $u_{2}$ which forgets the first marking and its left adjoint $\iota_{2}$.
%\end{itemize}

Just as in the marked case, we will a set of doubly-marked anodyne morphisms, the morphisms with the left lifting property with respect to bicartesian fibrations.

\begin{definition}
  The class of \defn{doubly-marked anodyne morphisms} is the saturated hull of the union of the following classes of morphisms.
  \begin{enumerate}
    \item[(1)] For each $0  < i < n$, the inner horn inclusions
      \begin{equation*}
        (\Lambda^{n}_{i})^{(\flat, \flat)} \to (\Delta^{n})^{(\flat, \flat)}.
      \end{equation*}

    \item[(2)] For every $n > 0$, the inclusion
      \begin{equation*}
        (\Lambda^{n}_{0})^{(\mathcal{L}, \flat)} \hookrightarrow (\Delta^{n})^{(\mathcal{L}, \flat)},
      \end{equation*}
      where $\mathcal{L}$ denotes the set of all degenerate edges of $\Delta^{n}$ together with the edge $\Delta^{\{0, 1\}}$.

    \item[(2')] For every $n > 0$, the inclusion
      \begin{equation*}
        (\Lambda^{n}_{n})^{(\flat, \mathcal{R})} \hookrightarrow (\Delta^{n})^{(\flat, \mathcal{R})},
      \end{equation*}
      where $\mathcal{R}$ denotes the set of all degenerate edges of $\Delta^{n}$ together with the edge $\Delta^{\{n-1, n\}}$.

    \item[(3)] The inclusion
      \begin{equation*}
        (\Lambda^{2}_{1})^{(\sharp, \flat)} \coprod_{(\Lambda^{2}_{1})^{(\flat, \flat)}} (\Delta^{2})^{(\flat, \flat)} \to (\Delta^{2})^{(\sharp, \flat)}.
      \end{equation*}

    \item[(3')] The inclusion
      \begin{equation*}
        (\Lambda^{2}_{1})^{(\flat, \sharp)} \coprod_{(\Lambda^{2}_{1})^{(\flat, \flat)}} (\Delta^{2})^{(\flat, \flat)} \to (\Delta^{2})^{(\flat, \sharp)}.
      \end{equation*}

    \item[(4)] For every Kan complex $K$, the map
      \begin{equation*}
        K^{(\flat, \flat)} \to K^{(\sharp, \flat)}.
      \end{equation*}

    \item[(4')] For every Kan complex $K$, the map
      \begin{equation*}
        K^{(\flat, \flat)} \to K^{(\flat, \sharp)}.
      \end{equation*}
  \end{enumerate}
\end{definition}

\begin{example}
  \label{eg:bimarking_on_asd_delta1}
  Define a doubly-marked structure $(\asd(\Delta^{1}), \mathcal{E}, \mathcal{E}') = \asd(\Delta^{1})^{\heartsuit}$ on $\asd(\Delta^{1})$, where the morphism $01 \to 11$ is $\mathcal{E}$-marked, and the morphism $01 \to 00$ is $\mathcal{E}'$-marked. Denoting $\mathcal{E}$-marked morphisms with a $\bullet$ and $\mathcal{E}'$-marked morphisms with a $\circ$, we can draw this as follows.
  \begin{equation*}
    \begin{tikzcd}
      && 11
      \\
      & 01
      \arrow[ur, "\bullet" marking]
      \arrow[dl, "\circ" marking]
      \\
      00
    \end{tikzcd}
  \end{equation*}

  The inclusion $\{00\} = \asd(\Delta^{\{0\}}) \hookrightarrow \asd(\Delta^{1})^{\heartsuit}$ is doubly-marked anodyne: we can factor it
  \begin{equation*}
    \{00\} \hookrightarrow (\Delta^{1})^{(\flat, \sharp)} \hookrightarrow \asd(\Delta^{1})^{\heartsuit},
  \end{equation*}
  where the first inclusion is of the form (2') and the second is a pushout of a morphism of the form (2). We can draw this process as follows.
  \begin{equation*}
    \begin{tikzcd}
      \
      \\
      \
      \\
      00
    \end{tikzcd}
    \qquad\hookrightarrow\qquad
    \begin{tikzcd}
      & \
      \\
      & 01
      \arrow[dl, "\circ" marking]
      \\
      00
    \end{tikzcd}
    \qquad\hookrightarrow\qquad
    \begin{tikzcd}
      && 11
      \\
      & 01
      \arrow[ur, "\bullet" marking]
      \arrow[dl, "\circ" marking]
      \\
      00
    \end{tikzcd}
  \end{equation*}
\end{example}

\begin{example}
  \label{eg:bimarking_on_asd_delta2}
  Define a doubly-marked structure $\asd(\Delta^{2})^{\heartsuit}$ on $\asd(\Delta^{2})$, following the notation of \hyperref[eg:bimarking_on_asd_delta1]{Example~\ref*{eg:bimarking_on_asd_delta1}}, as follows.
  \begin{equation*}
    \begin{tikzcd}
      && 11
      \\
      & 01
      \arrow[ur, "\bullet" marking]
      \arrow[dl, "\circ" marking]
      && 12
      \arrow[ul]
      \arrow[dr]
      \\
      00
      && 02
      \arrow[ll]
      \arrow[rr]
      \arrow[uu]
      \arrow[ur, "\bullet" marking]
      \arrow[ul]
      && 22
    \end{tikzcd}
  \end{equation*}
  Note that the bimarking of \hyperref[eg:bimarking_on_asd_delta1]{Example~\ref*{eg:bimarking_on_asd_delta1}} is the restriction of $\asd(\Delta^{2})^{\heartsuit}$ to $\asd(\Delta^{\{0, 1\}})$. Denote the restriction of the bimarking $\asd(\Delta^{2})^{\heartsuit}$ to $\asd(\Delta^{\{0, 2\}})$ by $\asd(\Delta^{\{0, 2\}})^{\heartsuit}$; this agrees with the $(\flat,\flat)$-marking.

  The inclusion $\asd(\Lambda^{2}_{0})^{\heartsuit} \hookrightarrow \asd(\Delta^{2})^{\heartsuit}$ is doubly-marked anodyne. To see this, note the following factorization.
\end{example}

\begin{proposition}
  \label{prop:rlp_doubly-marked_anodyne}
  A map $p\colon (X, \mathcal{E}_{X}, \mathcal{E}_{X}') \to (S, \mathcal{E}_{S}, \mathcal{E}_{S}')$ of doubly-marked simplicial sets has the right lifting property with respect to doubly-marked anodyne morphisms if and only if the following conditions are satisifed.
  \begin{enumerate}
    \item[(A)] The map $p$ is an inner fibration of simplicial sets.

    \item[(B)] An edge $e$ of $X$ is $\mathcal{E}_{X}$-marked if and only if $p(e)$ is $\mathcal{E}_{S}$-marked and $e$ is $p$-cocartesian.

    \item[(B')] An edge $e$ of $X$ is $\mathcal{E}'_{X}$-marked if and only if $p(e)$ is $\mathcal{E}'_{S}$-marked and $e$ is $p$-cartesian.

    \item[(C)] For every object $y$ of $X$ and every $\mathcal{E}_{S}$-marked edge $\bar{e}\colon \bar{x} \to p(y)$ in $S$, there exists a $\mathcal{E}_{X}$ marked edge $e\colon x \to y$ of $X$ with $p(e) = \bar{e}$.

    \item[(C')] For every object $y$ of $X$ and every $\mathcal{E}'_{S}$-marked edge $\bar{e}\colon \bar{x} \to p(y)$ in $S$, there exists a $\mathcal{E}'_{X}$ marked edge $e\colon x \to y$ of $X$ with $p(e) = \bar{e}$.
  \end{enumerate}
\end{proposition}
\begin{proof}
  By \cite[Prop.\ 3.1.1.6]{highertopostheory}, (A), (B) and (C) are equivalent to (1), (2), and (3). By its dual, (A), (B') and (C') are equivalent to (1), (2'), and (3')
\end{proof}

We would like to define cofibrations of doubly-marked simplicial sets to be maps of doubly-marked simplicial sets whose underlying map of simplicial sets is a monomorphism, and then show that the class of doubly-marked anodyne maps is stable under smash products with arbitrary cofibrations. Unfortunately, this turns out not to be quite true; the candidate class of cofibrations described above is generated by the following classes of maps.
\begin{itemize}
  \item[(I)] Boundary fillings $(\partial \Delta^{n})^{(\flat, \flat)} \to (\Delta^{n})^{(\flat, \flat)}$.

  \item[(II)] Markings $(\Delta^{1})^{(\flat, \flat)} \to (\Delta^{1})^{(\sharp, \flat)}$.

  \item[(III)] Markings $(\Delta^{1})^{(\flat, \flat)} \to (\Delta^{1})^{(\flat, \sharp)}$.
\end{itemize}
There is nothing that tells us, for example, that the smash product of a doubly-marked anodyne map of type $(2)$ with a cofibration of type (III) should be doubly-marked anodyne. Denoting arrows with the first marking using a $\bullet$ and the second using a $\circ$, this amounts, in the case $n = 0$, to the statement that the map
\begin{equation*}
  \begin{tikzcd}
    \cdot
    \arrow[r]
    \arrow[d, "\bullet" marking]
    \arrow[dr]
    & \cdot
    \arrow[d, "\bullet" marking]
    \\
    \cdot
    \arrow[r, "\circ" marking]
    & \cdot
  \end{tikzcd}
  \quad \longrightarrow \quad
  \begin{tikzcd}
    \cdot
    \arrow[r, "\circ" marking]
    \arrow[d, "\bullet" marking]
    \arrow[dr]
    & \cdot
    \arrow[d, "\bullet" marking]
    \\
    \cdot
    \arrow[r, "\circ" marking]
    & \cdot
  \end{tikzcd}
\end{equation*}
should be doubly-marked anodyne, which it isn't. However, we have the following weaker statement.

\begin{lemma}
  \label{lemma:smash_product_of_doubly-marked_anodyne_and_monic_is_doubly-marked_anodyne}
  The class of doubly-marked anodyne maps in $\SSet^{++}$ is stable under smash products with flat monomorphisms, i.e.\ morphisms $A^{(\flat, \flat)} \to B^{(\flat, \flat)}$ such that the underlying morphism of simplicial sets $A \to B$ is a monomorphism. That is, if $f\colon X \to Y$ is doubly-marked anodyne and $A \to B$ is a monomorphism of simplicial sets, then
  \begin{equation*}
    (X \times B^{(\flat, \flat)}) \coprod_{X \times A^{(\flat, \flat)}} (Y \times A^{(\flat, \flat)}) \to Y \times B^{(\flat, \flat)}
  \end{equation*}
  is doubly-marked anodyne.
\end{lemma}
\begin{proof}
  It suffices to show that for any flat boundary inclusion $(\partial \Delta^{n})^{(\flat, \flat)} \to (\Delta^{n})^{(\flat, \flat)}$ and any generating doubly-marked anodyne morphism $X \to Y$, the map
  \begin{equation*}
    (X \times (\Delta^{n})^{(\flat, \flat)}) \coprod_{X \times (\partial \Delta^{n})^{(\flat, \flat)}} (Y \times (\partial\Delta^{n})^{(\flat, \flat)}) \to Y \times (\Delta^{n})^{(\flat, \flat)}
  \end{equation*}
  is doubly-marked anodyne. If $X \to Y$ belongs to one of the classes (1), (2'), (3'), or (4'), then this is true by the arguments of \cite[Prop.\ 3.1.2.3]{highertopostheory}. If $X \to Y$ belongs to one of the classes (1), (2), (3), or (4), then it is true by the dual arguments.
\end{proof}

\begin{definition}
  For any doubly-marked simplicial sets $X$, $Y$, define a simplicial set $\Map^{(\flat, \flat)}(X, Y)$ by the following universal property: for any simplicial set $Z$, there is a bijection
  \begin{equation*}
    \Hom_{\SSet}(Z, \Map^{(\flat, \flat)}(X, Y)) \cong \Hom_{\SSet^{++}}(Z^{(\flat, \flat)} \times X, Y).
  \end{equation*}
\end{definition}

\begin{proposition}
  \label{proposition:doubly-marked_anodyne_homotopy_pullback}
  Let $p\colon \category{C} \to \category{D}$ be a bicartesian fibration of quasicategories, and denote by $\category{C}^{\natural} \to \category{D}^{(\sharp, \sharp)}$ the associated map of doubly-marked simplicial sets as in \hyperref[eg:bicartesian_marking]{Example~\ref*{eg:bicartesian_marking}}. Let $X \to Y$ be any doubly-marked anodyne map of simplicial sets. Then the square
  \begin{equation*}
    \begin{tikzcd}
      \Fun^{(\flat, \flat)}(Y, \category{C}\nat)^{\simeq}
      \arrow[r]
      \arrow[d]
      & \Fun^{(\flat, \flat)}(X, \category{C}\nat)^{\simeq}
      \arrow[d]
      \\
      \Fun^{(\flat, \flat)}(Y, \category{D}^{(\sharp, \sharp)})^{\simeq}
      \arrow[r]
      & \Fun^{(\flat, \flat)}(X, \category{D}^{(\sharp, \sharp)})^{\simeq}
    \end{tikzcd}
  \end{equation*}
  is a homotopy pullback in the Kan model structure.
\end{proposition}
\begin{proof}
  First, we show that the right-hand map is a Kan fibration. In fact, the underlying map
  \begin{equation*}
    \Fun^{(\flat, \flat)}(X, \category{C}^{\natural}) \to \Fun^{(\flat, \flat)}(X, \category{D}^{(\sharp, \sharp)})
  \end{equation*}
  is a trivial Kan fibration, since by \hyperref[lemma:smash_product_of_doubly-marked_anodyne_and_monic_is_doubly-marked_anodyne]{Lemma~\ref*{lemma:smash_product_of_doubly-marked_anodyne_and_monic_is_doubly-marked_anodyne}} together with \hyperref[prop:rlp_doubly-marked_anodyne]{Proposition~\ref*{prop:rlp_doubly-marked_anodyne}} we can solve the necessary lifting problems. This, together with the fact that each of the objects is a Kan complex, implies that in order to show that the above square is homotopy pullback it suffices to check that the map
  \begin{equation*}
    \Fun^{(\flat, \flat)}(Y, \category{C}\nat)^{\simeq} \to \Fun^{(\flat, \flat)}(X, \category{C}\nat)^{\simeq} \times_{\Fun^{(\flat, \flat)}(Y, \category{D}^{(\sharp, \sharp)})^{\simeq}} \Fun^{(\flat, \flat)}(X, \category{D}^{(\sharp, \sharp)})^{\simeq}
  \end{equation*}
  is a trivial Kan fibration. Since the functor $(-)^{\simeq}$ is a right adjoint it preserves limits, so it again suffices to show that the underlying map
  \begin{equation*}
    \Fun^{(\flat, \flat)}(Y, \category{C}\nat) \to \Fun^{(\flat, \flat)}(X, \category{C}\nat) \times_{\Fun^{(\flat, \flat)}(Y, \category{D}^{(\sharp, \sharp)})} \Fun^{(\flat, \flat)}(X, \category{D}^{(\sharp, \sharp)})
  \end{equation*}
  is a trivial fibration, which follows from \hyperref[lemma:smash_product_of_doubly-marked_anodyne_and_monic_is_doubly-marked_anodyne]{Lemma~\ref*{lemma:smash_product_of_doubly-marked_anodyne_and_monic_is_doubly-marked_anodyne}}.
\end{proof}

\subsection{Cartesian morphisms}
\label{ssc:cartesian_morphisms}

In this section, we show that we really have identified the cocartesian morphisms correctly. We will first show that morphisms of the form
\begin{equation}
  \label{eq:form_of_p_cocartesian_morphisms}
  \begin{tikzcd}
    & y
    \arrow[dl, "\circ" marking]
    \arrow[dr, "\bullet" marking]
    \\
    x
    && x'
  \end{tikzcd}.
\end{equation}
are $p$-cocartesian.

\begin{proposition}
  \label{prop:form_of_cocartesian_morphisms_in_spans}
  Let $\pi\colon \category{C} \to \category{D}$ be a bicartesian fibration of quasicategories which preserves pullbacks and satisfies the condition of \hyperref[thm:span_of_bicartesian_fibration_is_bicartesian_fibration]{Theorem~\ref*{thm:span_of_bicartesian_fibration_is_bicartesian_fibration}}, and let
  \begin{equation*}
    p\colon \Span(\category{C}) \to \Span(\category{D})
  \end{equation*}
  be the corresponding map between complete Segal spaces of spans. If a morphism in $\Span(\category{C})$ is of the form
  \begin{equation*}
    \begin{tikzcd}
      & y
      \arrow[dl, "\circ" marking]
      \arrow[dr, "\bullet" marking]
      \\
      x
      && x'
    \end{tikzcd},
  \end{equation*}
  where the morphism marked with a $\circ$ is $\pi$-cartesian and the morphism marked with a $\bullet$ is $\pi$-cocartesian, then it is $p$-cocartesian.
\end{proposition}
\begin{proof}
  In order to show that a morphism $e\colon x \leftarrow y \rightarrow x'$ of the form given in \hyperref[prop:form_of_cocartesian_morphisms_in_spans]{Proposition~\ref*{prop:form_of_cocartesian_morphisms_in_spans}} are cocartesian, we have to show that
  %the square
  %\begin{equation*}
  %  \begin{tikzcd}
  %    \Span(\category{C})_{2} \times_{\Span(\category{C})_{\{0, 1\}}} \{e\}
  %    \arrow[r]
  %    \arrow[d]
  %    & \Span(\category{C})_{\{0, 2\}} \times_{\Span(\category{C})_{\{0\}}} \{x\}
  %    \arrow[d]
  %    \\
  %    \Span(\category{D})_{2} \times_{\Span(\category{D})_{\{0, 1\}}} \{\pi e\}
  %    \arrow[r]
  %    & \Span(\category{D})_{\{0, 2\}} \times_{\Span(\category{D})_{\{0\}}} \{\pi x\}
  %  \end{tikzcd}
  %\end{equation*}
  %is homotopy pullback. Expanding, we have to show that
  the square
  \begin{equation*}
    \begin{tikzcd}
      \Fun^{\Cart}(\asd(\Delta^{2}), \category{C})^{\simeq} \times_{\Fun(\asd(\Delta^{\{0, 1\}}), \category{C})^{\simeq}} \{e\}
      \arrow[r]
      \arrow[d]
      & \Fun(\asd(\Lambda^{2}_{0}), \category{C})^{\simeq} \times_{\Fun(\asd(\Delta^{\{0, 1\}}), \category{C})^{\simeq}} \{e\}
      \arrow[d]
      \\
      \Fun^{\Cart}(\asd(\Delta^{2}), \category{D})^{\simeq} \times_{\Fun(\asd(\Delta^{\{0, 1\}}), \category{D})^{\simeq}} \{\pi e\}
      \arrow[r]
      & \Fun(\asd(\Lambda^{2}_{0}), \category{D})^{\simeq} \times_{\Fun(\asd(\Delta^{\{0, 1\}}), \category{D})^{\simeq}} \{\pi e\}
    \end{tikzcd}
  \end{equation*}
  is homotopy pullback.

  Recall the doubly-marked structure $\asd(\Delta^{2})^{\heartsuit} = (\asd(\Delta^{2}), \mathcal{E}, \mathcal{E}')$ on $\asd(\Delta^{2})$ of \hyperref[eg:bimarking_on_asd_delta2]{Example~\ref*{eg:bimarking_on_asd_delta2}}, reproduced below, where the nondegenerate edges in $\mathcal{E}$ are distinguished with a $\bullet$, and the nondegenerate edges in $\mathcal{E}'$ are distinguished with a $\circ$.
  \begin{equation*}
    \begin{tikzcd}
      && 11
      \\
      & 01
      \arrow[ur, "\bullet" marking]
      \arrow[dl, "\circ" marking]
      && 12
      \arrow[ul]
      \arrow[dr]
      \\
      00
      && 02
      \arrow[ll]
      \arrow[rr]
      \arrow[uu]
      \arrow[ur, "\bullet" marking]
      \arrow[ul]
      && 22
    \end{tikzcd}
  \end{equation*}
  Denote the induced doubly-marked structure on $\asd(\Lambda^{\{2, 1\}})$ also with a heart.

  We now note that we can decompose the above square into two squares
  \begin{equation*}
    \begin{tikzcd}[column sep=tiny]
      \Fun^{\Cart}(\asd(\Delta^{2}), \category{C})^{\simeq} \times_{\Fun(\asd(\Delta^{\{0, 1\}}), \category{C})^{\simeq}} \{e\}
      \arrow[r]
      \arrow[d]
      & \Fun^{(\flat, \flat)}(\asd(\Delta^{2})^{\heartsuit}, \category{C}^{\natural})^{\simeq} \times_{\Fun^{(\flat, \flat)}(\asd(\Delta^{\{0, 1\}})^{\heartsuit}, \category{C}^{\natural})^{\simeq}} \{e\}
      \arrow[d]
      \\
      \Fun^{\Cart}(\asd(\Delta^{2}), \category{D})^{\simeq} \times_{\Fun(\asd(\Delta^{\{0, 1\}}), \category{D})^{\simeq}} \{\pi e\}
      \arrow[r]
      & \Fun^{(\flat, \flat)}(\asd(\Delta^{2})^{\heartsuit}, \category{D}^{(\sharp, \sharp)})^{\simeq} \times_{\Fun^{(\flat, \flat)}(\asd(\Delta^{\{0, 1\}})^{\heartsuit}, \category{D}^{(\sharp, \sharp)})^{\simeq}} \{\pi e\}
    \end{tikzcd}
  \end{equation*}
  and
  \begin{equation*}
    \begin{tikzcd}[column sep=tiny]
      \Fun^{(\flat, \flat)}(\asd(\Delta^{2})^{\heartsuit}, \category{C}^{\natural})^{\simeq} \times_{\Fun^{(\flat, \flat)}(\asd(\Delta^{\{0, 1\}})^{\heartsuit}, \category{C}^{\natural})^{\simeq}} \{e\}
      \arrow[r]
      \arrow[d]
      & \Fun(\asd(\Lambda^{2}_{0}), \category{C})^{\simeq} \times_{\Fun(\asd(\Delta^{\{0, 1\}}), \category{C})^{\simeq}} \{e\}
      \arrow[d]
      \\
      \Fun^{(\flat, \flat)}(\asd(\Delta^{2})^{\heartsuit}, \category{D}^{(\sharp, \sharp)})^{\simeq} \times_{\Fun^{(\flat, \flat)}(\asd(\Delta^{\{0, 1\}})^{\heartsuit}, \category{D}^{(\sharp, \sharp)})^{\simeq}} \{\pi e\}
      \arrow[r]
      & \Fun(\asd(\Lambda^{2}_{0}), \category{D})^{\simeq} \times_{\Fun(\asd(\Delta^{\{0, 1\}}), \category{D})^{\simeq}} \{\pi e\}
    \end{tikzcd}.
  \end{equation*}
  The first square is a homotopy pullback because the bottom morphism is a full subcategory inclusion of connected components, and the fiber over a connected component corresponding to Cartesian functors $\asd(\Delta^{2}) \to \category{D}$ is consists precisely of Cartesian functors $\asd(\Delta^{2}) \to \category{C}$ by the condition of \hyperref[thm:span_of_bicartesian_fibration_is_bicartesian_fibration]{Theorem~\ref*{thm:span_of_bicartesian_fibration_is_bicartesian_fibration}}.

  Therefore, we need to show that the second square is homotopy pullback. For this it suffices to show that the square
  \begin{equation*}
    \begin{tikzcd}
      \Fun^{(\flat, \flat)}(\asd(\Delta^{2})^{\heartsuit}, \category{C}^{\natural})^{\simeq}
      \arrow[r]
      \arrow[d]
      & \Fun(\asd(\Lambda^{2}_{0}), \category{C})^{\simeq}
      \arrow[d]
      \\
      \Fun^{(\flat, \flat)}(\asd(\Delta^{2})^{\heartsuit}, \category{D}^{(\sharp, \sharp)})^{\simeq}
      \arrow[r]
      & \Fun(\asd(\Lambda^{2}_{0}), \category{D})^{\simeq}
    \end{tikzcd}.
  \end{equation*}
  is homotopy pullback. But that this is of the form of the square in \hyperref[proposition:doubly-marked_anodyne_homotopy_pullback]{Proposition~\ref*{proposition:doubly-marked_anodyne_homotopy_pullback}}, with $X \to Y = \asd(\Lambda^{2}_{0})^{\heartsuit} \to \asd(\Delta^{2})^{\heartsuit}$, which we saw in \hyperref[eg:bimarking_on_asd_delta2]{Example~\ref*{eg:bimarking_on_asd_delta2}} was doubly-marked anodyne.
\end{proof}

\begin{theorem}
  Let $\category{C}$ and $\category{D}$ be quasicategories with pullbacks, and let $\pi\colon \category{C} \to \category{D}$ be a bicartesian fibration which sends pullbacks to pullbacks. Then the map
  \begin{equation*}
    \Span(\category{C})/\Delta^{0} \to \Span(\category{D})/\Delta^{0}
  \end{equation*}
  is a cocartesian fibration (hence a bicartesian fibration) between quasicategories.
\end{theorem}
\begin{proof}
  Take $\Span(\category{C})$ to have the cocartesian marking, which we have denoted by $\Span(C)^{\natural}$. In order to fulfill the requirements of \hyperref[cor:cocart_fib_between_css_gives_cocart_fib_of_quasicats]{Corollary~\ref*{cor:cocart_fib_between_css_gives_cocart_fib_of_quasicats}}, it suffices to show that this marking respects path components. But this is clear: a 1-simplex $g \to g'$ in $\Span(\category{C})^{\natural}_{1}$ between morphisms
  \begin{equation*}
    g\colon X \leftarrow Y \to Z \qquad\text{and}\qquad g'\colon X' \leftarrow Y' \to Z'
  \end{equation*}
  looks as follows.
  \begin{equation*}
    \begin{tikzcd}
      X
      \arrow[d, "\simeq"]
      & Y
      \arrow[l, swap, "g_{0}"]
      \arrow[r, "g_{1}"]
      \arrow[d, "\simeq"]
      & Z
      \arrow[d, "\simeq"]
      \\
      X'
      & Y'
      \arrow[l, swap, "g'_{0}"]
      \arrow[r, "g'_{1}"]
      & Z'
    \end{tikzcd}
  \end{equation*}
  Since equivalences are both cartesian and cocartesian, it is clear that $g$ is cocartesian if and only if $g'$ is cocartesian.
\end{proof}


\begin{appendix}
  \section{Background results}
  \label{sec:background}

  \subsection{Results from J+T}
  \label{ssc:results_from_j_t}

  \subsubsection{Bisimplicial sets}
  \label{sss:bisimplicial_sets}

  We will denote the simplex category by $\D$, and the category of functors $\D\op \to \Set$ by $\SSet$. This is the category of \emph{simplicial sets.}

  The category $\SSet$ carries two model structures of which we will make frequent use:
  \begin{itemize}
    \item The \emph{Kan} model structure, which has the following description.
      \begin{itemize}
        \item The fibrations are the Kan fibrations.

        \item The cofibrations are the monomorphisms.

        \item The weak equivalences are weak homotopy equivalences.
      \end{itemize}

    \item The \emph{Joyal} model structure. We will not give a complete description, referring the reader to \cite[Sec.\ 2.2.5]{highertopostheory}. We will make use of the following properties.
      \begin{itemize}
        \item The cofibrations are monomorphisms.

        \item The fibrant objects are quasicategories, and the fibrations between fibrant objects are isofibrations, i.e.\ inner fibrations with lifts of equivalences (cf \cite[Cor.\ 2.6.5]{highertopostheory}).
      \end{itemize}
  \end{itemize}

  The category $\D\op$ has a Reedy structure, which gives us, together with the Kan model structure, a model structure on the category $\Fun(\D\op, \SSet)$ with the following properties.
  \begin{itemize}
    \item The cofibrations are monomorphisms.

    \item The weak equivalences are level-wise weak homotopy equivalences.

    \item We will call the fibrations \emph{Reedy fibrations,} and will come to them in a moment.
  \end{itemize}



  Bisimplicial sets can be equivalently defined in two equivalent ways:
  \begin{itemize}
    \item As functors $\D\op \to \SSet$

    \item As functors $(\D\op)^{2} \to \Set$.
  \end{itemize}
  In the former case, we think of a bisimplicial set $X$ as an $\N$-indexed collection of simplicial sets $X_{n}$; in the latter, we think of a bisimplicial set as an $\N \times \N$-indexed collection of sets $X_{mn}$. It will be important to keep both of these points of view in mind. For this reason, we fix the following convention: we imagine a bisimplicial set $X$ as a collection of sets, each located at a lattice points of the first quadrant, where the first coordinate increases in the $x$-direction and the second coordinate increases in the $y$-direction. Thus, the $m$th row of $X$ is the simplicial set $X_{\bullet m}$, and the $n$th column of $X$ is the simplicial set $X_{n, \bullet}$. When we think of bisimplicial sets as $\N$-indexed collections of simplicial sets $X_{n}$, we mean 

  \subsubsection{The box functor}
  \label{sss:the_box_functor}

  In this section, we recall some key results from \cite{qcats_vs_segal_spaces}. We refer readers there for more information.

  Let $\odot\colon \mathcal{E}_{2} \times \category{E}_{2} \to \category{E}_{3}$ be a functor. Following \cite{qcats_vs_segal_spaces}, we will say that $\odot$ is \emph{divisible on the left} if for each $A \in \category{E}_{1}$, the functor $A \odot -$ admits a right adjoint $A \backslash -$, and \emph{divisible on the right} if for for each $B \in \category{E}_{2}$, the functor $- \odot B$ admits a right adjoint $- / B$. In the case that $\odot$ is divisible on the left, we get a of two variables
  \begin{equation*}
    - / -\colon \category{E}_{3} \times \category{E}_{2}\op \to \category{E}_{1}.
  \end{equation*}
  In the case that $\odot$ is divisible on the right, we get a functor
  \begin{equation*}
    - \backslash -\colon \category{E}_{1}\op \times \category{E}_{3} \to \category{E}_{2}.
  \end{equation*}

  If $\odot$ is divisible on both sides, then it is equivalent to provide any of the maps
  \begin{equation*}
    A \square B \to X,\qquad A \to X / B,\qquad B \to A \backslash X,
  \end{equation*}
  implying that the functors $X / -$ and $- \backslash X$ are mutually right adjoint.

  If both $\category{E}_{1}$ and $\category{E}_{2}$ are finitely complete and $\category{E}_{3}$ is finitely cocomplete, then from a map $u\colon A \to A'$ in $\category{E}_{1}$, a map $v\colon B \to B'$ in $\category{E}_{2}$, and a map $f\colon X \to Y$ in $\category{E}_{3}$, we can build the following maps.
  \begin{itemize}
    \item From the square
      \begin{equation*}
        \begin{tikzcd}
          A \odot B
          \arrow[r]
          \arrow[d]
          & A' \odot B
          \arrow[d]
          \\
          A \odot B'
          \arrow[r]
          & A' \odot B'
        \end{tikzcd}
      \end{equation*}
      we get a map
      \begin{equation*}
        u \odot' v\colon A \odot B' \amalg_{A \odot B} A' \odot B \to A' \odot B'.
      \end{equation*}

    \item From the square
      \begin{equation*}
        \begin{tikzcd}
          A' \backslash X
          \arrow[r]
          \arrow[d]
          & A \backslash X
          \arrow[d]
          \\
          A' \backslash Y
          \arrow[r]
          & A \backslash Y
        \end{tikzcd}
      \end{equation*}
      we get a map
      \begin{equation*}
        \langle u \backslash f \rangle\colon A' \backslash X \to A \backslash X \times_{A \backslash Y} A' \backslash Y
      \end{equation*}

    \item From the square
      \begin{equation*}
        \begin{tikzcd}
          X / B'
          \arrow[r]
          \arrow[d]
          & X / B
          \arrow[d]
          \\
          Y / B'
          \arrow[r]
          & Y / B
        \end{tikzcd}
      \end{equation*}
      we get a map
      \begin{equation*}
    \langle f / v \rangle\colon X / B' \to X / B \times_{Y / B} Y / B'.
      \end{equation*}
  \end{itemize}

  In this case, the following are equivalent adjoint lifting problems:
  \begin{itemize}
    \item
      \begin{equation*}
        \begin{tikzcd}
          A \odot B' \amalg_{A \odot B} A' \odot B
          \arrow[r]
          \arrow[d]
          & X
          \arrow[d]
          \\
          A' \odot B'
          \arrow[r]
          \arrow[ur, dashed]
          & Y
        \end{tikzcd}.
      \end{equation*}

    \item
      \begin{equation*}
        \begin{tikzcd}
          A
          \arrow[r]
          \arrow[d]
          & X / B'
          \arrow[d]
          \\
          A'
          \arrow[r]
          \arrow[ur, dashed]
          & X / B \times_{Y / B} Y / B'
        \end{tikzcd}.
      \end{equation*}

    \item
      \begin{equation*}
        \begin{tikzcd}
          B
          \arrow[r]
          \arrow[d]
          & A' \backslash X 
          \arrow[d]
          \\
          B'
          \arrow[r]
          \arrow[ur, dashed]
          & A \backslash X \times_{A \backslash Y} A' \backslash Y
        \end{tikzcd}.
      \end{equation*}
  \end{itemize}

  \begin{definition}
    Define a functor $- \square -\colon \SSet \times \SSet \to \SSSet$ by the formula
    \begin{equation*}
      (X, Y) \mapsto X \square Y,\qquad (X \square Y)_{mn} = X_{m} \times Y_{n}.
    \end{equation*}
    We will call this functor the \defn{box functor.}
  \end{definition}

  We provide a partial proof of the following result because we will need to refer to it later.
  \begin{proposition}
    \label{prop:bijection_exhibiting_box_divisibility}
    The box functor is \emph{divisible on the left and on the right.} This means that for each simplicial set $A$, there is an
    \begin{equation*}
      A \square -\colon \SSet \leftrightarrow \SSSet : A \backslash -,
    \end{equation*}
    where an $n$-simplex of the simplicial set $A \backslash X$ is a map $A \square \Delta^{n} \to X$.

    There is a similar adjunction
    \begin{equation*}
      - \square A\colon \SSet \leftrightarrow \SSSet : - / A.
    \end{equation*}
  \end{proposition}
  \begin{proof}
    We prove the first; the second is identical. We do this by explicitly exhibiting a natural bijection
    \begin{equation*}
      \begin{tikzcd}
        \Hom_{\SSSet}(A \square B, X) \cong \Hom_{\SSet}(B, A \backslash X).
      \end{tikzcd}
    \end{equation*}

    Define a map
    \begin{equation*}
      \Phi\colon \Hom_{\SSSet}(A \square B, X) \to \Hom_{\SSet}(B, A \backslash X)
    \end{equation*}
    by sending a map $f\colon A \square B \to X$ to the map $\tilde{f}\colon B \to A \backslash X$ which sends an $n$-simplex $b \in B_{n}$ to the composition
    \begin{equation*}
      \begin{tikzcd}
        A \square \Delta^{n}
        \arrow[r, "{(\id, b)}"]
        & A \square B
        \arrow[r, "f"]
        & X
      \end{tikzcd}.
    \end{equation*}

    Before we define our inverse map, we need an intermediate result. Define a map $\ev\colon A \square (A \backslash X) \to X$ level-wise by taking $(a, \sigma) \in A_{m} \times (A \backslash X)_{n}$ to
    \begin{equation*}
      \sigma_{mn}(a, \id_{\Delta^{n}}) \in X_{mn}.
    \end{equation*}
    Then define a map
    \begin{equation*}
      \Psi\colon \Hom_{\SSet}(B, A \backslash X) \to \Hom_{\SSSet}(A \square B, X)
    \end{equation*}
    sending a map $g\colon B \to A \backslash X$ to the composition
    \begin{equation*}
      \begin{tikzcd}
        A \square B
        \arrow[r, "{(\id, g)}"]
        & A \square (A \backslash X)
        \arrow[r, "\ev"]
        & X
      \end{tikzcd}.
    \end{equation*}

    The maps $\Phi$ and $\Psi$ are mutually inverse, and provide the necessary natural bijection.

    The other bijection is given analogously, so we only fix notation which we will need later. We will call the mutually inverse maps
    \begin{equation*}
      \Phi'\colon \Hom_{\SSSet}(A \square B, X) \to \Hom_{\SSet}(A, X / B)
    \end{equation*}
    and
    \begin{equation*}
      \Psi'\colon \Hom_{\SSet}(A, X / B) \to \Hom_{\SSSet}(A \square B, X),
    \end{equation*}
    where in defining $\Psi'$ we use a map $\ev'\colon (X / B) \square B \to X$ sending
    \begin{equation*}
      (\phi\colon \Delta^{m} \square B \to X, b \in B_{n}) \mapsto \phi_{mn}(\id_{\Delta^{n}}, b).
    \end{equation*}
  \end{proof}

  \begin{definition}
    \label{def:reedy_fibration}
    Let $f\colon X \to Y$ be a map between bisimplicial sets. We will say that $f$ is a \defn{Reedy fibration} if either of the equivalent conditions holds.
    \begin{itemize}
      \item For each monomorphism $u\colon A \to A'$, the map $\langle u \backslash f \rangle$ is a Kan fibration.

      \item For each anodyne map $v\colon B \to B'$, the map $\langle f / v \rangle$ is a trivial Kan fibration.
    \end{itemize}
  \end{definition}
\end{appendix}

We will also make use of the following fact.

\begin{theorem}[\cite{qcats_vs_segal_spaces}]
  \label{thm:inner_fibration_between_quasicategories}
  If $f\colon X \to Y$ is a Reedy fibration between Segal spaces, then for any monomorphism of simplicial sets $v$, the map $\langle f / v \rangle$ is an inner fibration.
\end{theorem}

\end{document}
