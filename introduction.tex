\documentclass[main.tex]{subfiles}

\begin{document}

\section{Introduction}
\label{sec:introduction}

\subsection{Introduction}
\label{ssc:introduction}

In topology, one often studies spaces via algebraic invariants associated to them. One such invariant of particular interest is the \emph{$n$th homology} of a space $X$, denoted $H_{n}(X)$. Roughly speaking, $H_{n}(X)$ is the abelian group consisting of formal sums of certain maps from the standard $n$-simplex $\Delta^{n}$ into $X$, considered modulo some equivalence relation.

It turns out to be fruitful to generalize $n$th homology. Formal sums are nothing else but $\Z$-linear combinations, and one can ask what happens when $\Z$ is replaced by some other commutative ring $R$. The associated invariant is then known as \emph{$n$th homology with coefficients in $R$}. One can even allow the ring $R$ to vary along $X$, or allow $R$ to be not only a ring, but an object of some appropriate category $\category{C}$. The structure which keeps track of these changing coefficients is called a $\category{C}$-\emph{local system}. 

Classically speaking, when we say \emph{space,} we really mean \emph{topological space,} and $\category{C}$-local systems are modelled by locally-constant sheaves with values in $\category{C}$. In modern homotopy theory, one often takes \emph{space} to mean \emph{$\infty$-groupoid;} thought of in this way, a $\category{C}$-local system on a space $X$ is nothing more than a functor $X \to \category{C}$. This is the point of view that we will take.

With this point of view in mind, let $\category{C}$ be a cocomplete $\infty$-category. For any space $X$, denote by $\LS(\category{C})_{X}$ the $\infty$-category of $\category{C}$-local systems on $X$; that is, $\LS(\category{C})_{X} \simeq \Fun(X, \category{C})$. Recall that for any morphism $f\colon X \to Y$ of spaces, there are several associated functors between $\LS(\category{C})_{X}$ and $\LS(\category{C})_{Y}$. In particular:
\begin{itemize}
  \item The \emph{pullback functor} $f^{*}\colon \LS(\category{C})_{Y} \to \LS(\category{C})_{X}$ pulls back local systems on $Y$ to local systems on $X$, sending a local system $\mathcal{F}\colon Y \to \category{C}$ to the local system $f^{*}\mathcal{F}$ given by the composition
    \begin{equation*}
      \begin{tikzcd}
        X
        \arrow[r, "f"]
        & Y
        \arrow[r, "\mathcal{F}"]
        & \category{C}
      \end{tikzcd}.
    \end{equation*}

  \item The \emph{pushforward functor} $f_{!}\colon \LS(\category{C})_{X} \to \LS(\category{C})_{Y}$ pushes forward local systems via left Kan extension, sending a local system $\mathcal{G}\colon X \to \category{C}$ to the left Kan extension $f_{!}\mathcal{G}$.
    \begin{equation*}
      \begin{tikzcd}
        X
        \arrow[rr, "\mathcal{G}"]
        \arrow[dr, swap, "f"]
        && \category{C}
        \\
        & Y
        \arrow[ur, swap, "f_{!}G"]
      \end{tikzcd}
    \end{equation*}
\end{itemize}

This is a prototype of a common situation: one has a functor (say, of quasicategories) $F\colon \category{C} \to \category{D}$, which sends an object $c \in \category{C}$ to an object $F(c) \in \category{D}$, and a morphism $f\colon c \to c'$ in $\category{C}$ to a morphism $f_{!}\colon F(c) \to F(c')$ in $\category{D}$. One has additionally a `wrong-way' map, which creates from a morphism $f\colon c \to c'$ a morphism $f^{*}\colon F(c') \to F(c)$. In this case, from the data of a diagram in $\category{C}$ of the form
\begin{equation}
  \label{eq:span}
  \begin{tikzcd}
    & c'
    \arrow[dl, swap, "g"]
    \arrow[dr, "f"]
    \\
    c
    && c''
  \end{tikzcd},
\end{equation}
one can produce a morphism $F(c) \to F(c'')$ in $\category{D}$ via the composition
\begin{equation*}
  \begin{tikzcd}
    & F(c')
    \arrow[dr, "f_{!}"]
    \\
    F(c)
    \arrow[ur, "g^{*}"]
    \arrow[rr, swap, "f_{!} \circ g^{*}"]
    && F(c'')
  \end{tikzcd}.
\end{equation*}

The data of \hyperref[eq:span]{Diagram~\ref*{eq:span}} is known as a \emph{span} in $\category{C}$. It is natural to ask whether this construction can be extended to a functor $\Span(\category{C}) \to \category{D}$, where $\Span(\category{C})$ is an $\infty$-category whose objects are the same as the objects of $\category{C}$, and whose morphisms are spans in $\category{C}$. In defining $\Span(\category{C})$, it is necessary to specify a composition law for spans. There is a natural way of doing this: given two spans $X \leftarrow Y \rightarrow X'$ and $X' \leftarrow Y' \rightarrow X''$ in $\category{C}$, we define their composition to be the pullback $X \leftarrow Y \times_{X'} Y' \rightarrow X''$ as below.
\begin{equation*}
  \begin{tikzcd}
    && Y \times_{X'} Y'
    \arrow[dl, dashed, swap, "q'"]
    \arrow[dr, dashed, "f'"]
    \\
    & Y
    \arrow[dl, swap, "g"]
    \arrow[dr, "f"]
    && Y'
    \arrow[dl, swap, "q"]
    \arrow[dr, "p"]
    \\
    X
    && X'
    && X''
  \end{tikzcd}
\end{equation*}
In order for a functor $\Span(\category{C}) \to \category{D}$ to be well-defined, it will have to respect this composition law on $\Span(\category{C})$. That is, we must have for all such pullback diagrams an equivalence
\begin{equation*}
  (p \circ f')_{!} \circ (g \circ q')^{*} \simeq p_{!} \circ q^{*} \circ f_{!} \circ g^{*}.
\end{equation*}
This is equivalent to the simpler condition that for any pullback square as above we must have an equivalence
\begin{equation*}
  f'_{!} \circ (q')^{*} \simeq q^{*} \circ f_{!}.
\end{equation*}
This is know as the \emph{base change condtition,} or the \emph{Beck--Chevalley condition.}

We will define two $\infty$-categorical models for our category of spans in $\category{C}$: a Segal space $\SPAN(\category{C})$, and a quasicategory $\Span(\category{C})$. Let us for the moment concentrate on the quasicategorical model, although what we say would equally apply to the Segal space model. In defining the quasicategory $\Span(\category{C})$ of spans in $\category{C}$, it is necessary to specify a composition law for spans. There is a natural way of doing this: given two spans $X \leftarrow Y \rightarrow X'$ and $X' \leftarrow Y' \rightarrow X''$ in $\category{C}$, we define their composition to be the pullback $X \leftarrow Y \times_{X'} Y' \rightarrow X''$ as below.
\begin{equation*}
  \begin{tikzcd}
    && Y \times_{X'} Y'
    \arrow[dl, dashed, swap, "q'"]
    \arrow[dr, dashed, "f'"]
    \\
    & Y
    \arrow[dl, swap, "g"]
    \arrow[dr, "f"]
    && Y'
    \arrow[dl, swap, "q"]
    \arrow[dr, "p"]
    \\
    X
    && X'
    && X''
  \end{tikzcd}
\end{equation*}
In order for a functor $\Span(\category{C}) \to \category{D}$ to be well-defined, it will have to respect this composition law on $\Span(\category{C})$. That is, we must have for all such pullback diagrams an equivalence
\begin{equation*}
  (p \circ f')_{!} \circ (g \circ q')^{*} \simeq p_{!} \circ q^{*} \circ f_{!} \circ g^{*}.
\end{equation*}
This is equivalent to the simpler condition that for any pullback square as above we must have an equivalence
\begin{equation*}
  f'_{!} \circ (q')^{*} \simeq q^{*} \circ f_{!}.
\end{equation*}
This is know as the \emph{base change condtition,} or the \emph{Beck--Chevalley condition.}

This is the first part of a two-part paper. In this paper, we to provide a new proof of a theorem of Barwick \cite[Thm.~12.2]{spectralmackeyfunctors1}, which provides sufficient conditions for a functor of quasicategories $p\colon \category{C} \to \category{D}$ to yield a cocartesian fibration between $\infty$-categories of spans $\Span(p)\colon \Span(\category{C}) \to \Span(\category{D})$.

In the second part, we will use this result to construct a functor $\hat{r}\colon \Span(\category{S}) \to \ICat$, which sends a space $X$ to the category $\LS(\category{C})_{X}$ of $\category{C}$-local systems on $X$, and a morphism in $\Span(\S)$ represented by a span
\begin{equation*}
  \begin{tikzcd}
    & Y
    \arrow[dl, swap, "g"]
    \arrow[dr, "f"]
    \\
    X
    && X'
  \end{tikzcd}
\end{equation*}
to the functor $f_{!} \circ g^{*}\colon \LS(\category{C})_{X} \to \LS(\category{C})_{X'}$. We further show that our functor is lax monoidal with respect to a monoidal structure on $\Span(\S)$ induced by the cartesian structure on $\S$, and the cartesian structure on $\ICat$.

We begin by providing a new proof of the theorem of Barwick mentioned above. Barwick's proof is explicit, constructing the necessary horn fillings by enumerating the necessary simplices, and arguing one-by-one why each filling is possible. This is an impressive feat of simplicial combinatorics, but provides little intuition for why the result might be true.

Our proof is more homotopy-theoretic in character, relying on the fact that for any quasicategory $\category{C}$ with pullbacks, the $\infty$-category of spans in $\category{C}$ has a a natural incarnation as a complete Segal space $\SPAN(\category{C})$; the quasicategory $\Span(\category{C})$ is then the `first row' of this complete Segal space. We define a notion of cocartesian fibration between complete Segal spaces, and show that any such cocartesian fibration gives a cocartesian fibration between first rows. The readily available homotopical data in complete Segal spaces allow us to define cocartesian morphisms purely via a condition on 2-simplices, where the combinatorics of horn filling in categories of spans is more manageable.

\subsection{Relation to previous work}
\label{ssc:relation_to_previous_work}

The main result of this paper is a new, simplified proof of the main theorem of \cite{spectralmackeyfunctors1}, leveraging the theory of cocartesian morphisms in Segal spaces.

Our definition of a cocartesian fibration between Segal spaces is not new, although the form in which it is presented is original. The definition was first written down by De Brito in \cite{2016arXiv160500706B}, and expanded by Rasekh in \cite{rasekhcartesianfibrations}, and later in \cite{2021arXiv210205192R}. Here, Rasekh proves the existence and equivalence of several model structures for cocartesian fibrations of Segal spaces, including one in terms of marked bisimplicial sets (this did not exist when this paper was written). 

In our treatment, also in terms of marked bisimplicial sets, we have tried to balance explicitness of computations, not shying away from combinatorial calculations, with simplicity, not proving results in more generality than we need when we feel it does not lead to greater conceptual clarity. Thus, we have not constructed a model structure of surrounding our definition, and we treat only the case where both the total and base space are fibrant, both of which are done in \cite{2021arXiv210205192R}. We feel that the more calculational nature of our treatment leads to a greater interoperability with the existing theory of cocartesian fibrations between quasicategories as found in \cite{highertopostheory}.

\subsection{Outline}
\label{ssc:outline}

This work consists of three sections. In \hyperref[sec:cocartesian_fibrations_between_complete_segal_spaces]{Section~\ref*{sec:cocartesian_fibrations_between_complete_segal_spaces}}, we explore cocartesian fibrations between Segal spaces. After a review in \hyperref[ssc:a_review_of_bisimplicial_sets]{Section~\ref*{ssc:a_review_of_bisimplicial_sets}} of some material in \cite{qcats_vs_segal_spaces}, most notably the box product $- \square -$ and its adjoints, we define marked bisimplicial sets in \hyperref[ssc:marked_bisimplicial_sets]{Section~\ref*{ssc:marked_bisimplicial_sets}}. We then define a marked version of the box functor, and prove some results analogous to the unmarked case. In \hyperref[ssc:simplicial_technology]{Section~\ref*{ssc:simplicial_technology}}, we prove some technical lemmas about simplicial sets with certain restrictions on individual morphisms. The main result is a lemma which allows us to translate a condition on marked bisimplicial sets for a `pointwise condition' involving unmarked bisimplicial sets.

In \hyperref[ssc:cocartesian_fibrations]{Section~\ref*{ssc:cocartesian_fibrations}}, we define the notion of a cocartesian morphism via a condition on left horn filling of $2$-simplices, and define a cocartesian fibration between complete Segal spaces to be a Reedy fibration admitting cocartesian lifts. We show that this implies all higher left horn filling conditions, and use this to show that any cocartesian fibration between complete segal spaces yields a cocartesian fibration (in the sense of quasicategories) between first rows.

In \hyperref[sec:segal_spaces_of_spans]{Section~\ref*{sec:segal_spaces_of_spans}}, we use these results to prove the theorem of Barwick mentioned above, originally published in \cite[Thm.~12.2]{spectralmackeyfunctors1}. The proof leverages the fact, proven in \hyperref[sec:cocartesian_fibrations_between_complete_segal_spaces]{Section~\ref*{sec:cocartesian_fibrations_between_complete_segal_spaces}} that, when checking that a morphism in a Segal space is cocartesian relative to some Reedy fibration, it suffices to check that the lowest lifting problem $\Lambda^{2}_{1} \hookrightarrow \Delta^{2}$ has a contractible space of solutions, which in the setting of categories of spans is combinatorially tractible.


\subsection{Acknowledgements}
\label{ssc:acknowledgements}

We would like to thank his advisor, Tobias Dyckerhoff, for his patience and guidance. We would also like to thank Fernando Abell{\'a}n Garc{\'i}a for many helpful conversations.

\end{document}
