\documentclass[main.tex]{subfiles}

\begin{document}

\chapter{Segal spaces and presheaves}
\label{ch:segal_spaces_and_presheaves}

\section{Prerequisites}
\label{sec:prerequisites}

\subsection{Reedy model structure}
\label{ssc:reedy_model_structure}

\begin{definition}
  A \defn{Reedy category} is a category $\category{R}$ equipped with the following data:
  \begin{enumerate}
    \item A wide subcategory\footnote{A subcategory is \emph{wide} if it contains all objects.} $\category{R}_{+}$;

    \item A wide subcategory $\category{R}_{-}$; and

    \item A function $d\colon \ob(\category{R}) \to \N$ called the \emph{degree};
  \end{enumerate}
  subject to the following constraints:
  \begin{enumerate}
    \item Every non-identity morphism in $\category{R}_{+}$ raises degree;

    \item Every non-identity morphism in $\category{R}_{-}$ lowers degree; and

    \item Every morphism $f$ in $\category{R}$ has a unique factorization $f = f_{+} \circ f_{-}$, with $f_{-} \in \category{R}_{-}$ and $f_{+} \in \category{R}_{+}$.
  \end{enumerate}
\end{definition}

\begin{example}
  The simplex category $\Delta$ is a Reedy category with $d([n]) = n$, and $\Delta_{+}$ and $\Delta_{-}$ the wide subcatetegories on injections and surjections respectively.
\end{example}

\begin{example}
  \label{eg:opposite_reedy_structure}
  The opposite of any Reedy category is again a Reedy category with the roles of $\category{R}_{+}$ and $\category{R}_{-}$ reversed. In particular, $\Delta\op$ is a Reedy category with $\Delta\op_{+}$ the wide subcategory on surjections and $\Delta\op_{-}$ the wide subcategory on injections.
\end{example}

\begin{definition}
  Let $\category{R}$ be a Reedy category, and let $\category{C}$ be a model category. Let $X\colon \category{R} \to \category{C}$, and $r \in \category{R}$.

  \begin{itemize}
    \item The \defn{latching object} of $r$ is the object
      \begin{equation*}
        L_{r}X = \colim \left[ (\category{R}_{+} \downarrow r)' \to \category{R} \to \category{C} \right],
      \end{equation*}
      where the category $(\category{R}_{+} \downarrow r)'$ is the full subcategory of $(\category{R}_{+} \downarrow \category{R})$ containing all objects except the identity.

    \item The \defn{matching object} of $r$ is the object
      \begin{equation*}
        M_{r}X = \lim \left[ (r \downarrow \category{R}_{-})' \to \category{R} \to \category{C} \right],
      \end{equation*}
      where the category $(r \downarrow \category{R}_{-})'$ is the full subcategory of $(r \downarrow \category{R}_{-})$ containing all objects except the identity.
  \end{itemize}
\end{definition}

Note that because $X_{r}$ gives a cocone under $(\category{R}_{+} \downarrow r) \to \category{R} \to \category{C}$, we get a map $L_{r}X \to X_{r}$. Dually, we get a map $X_{r} \to M_{r}X$.

\begin{example}
  Take $\category{R} = \Delta\op$ with the Reedy structure defined in \hyperref[eg:opposite_reedy_structure]{Example~\ref*{eg:opposite_reedy_structure}}, and let $X$ be a simplicial set. The latching object $L_{n}X$ is the coequalizer of the diagram
  \begin{equation*}
    \begin{tikzcd}
      \displaystyle\coprod\limits_{\substack{f \in \text{Morph}(\Delta\op_{+} \downarrow [n])' \\ [n] \twoheadrightarrow [m] \overset{f}{\twoheadrightarrow} [m']}} X_{m}
      \arrow[r, shift left]
      \arrow[r, shift right]
      & \displaystyle\coprod\limits_{([m], g) \in \ob(\Delta\op_{+} \downarrow [n])'} X_{m}.
    \end{tikzcd}
  \end{equation*}
  Manipulating, one sees that this is the precisely the set of degenerate $n$-simplices.

  Dually, the matching object $M_{n}X$ is the hom-set $\Hom_{\SSet}(\partial \Delta^{n}, X)$.
\end{example}

Note that $L_{n}$ and $M_{n}$ are both functors $[\category{R}^{> n}, \category{C}]$, where $\category{R}^{> n}$ is the full subcategory of $\category{R}$ on objects $r$ such that $d(r) > n$.

\begin{theorem}
  Let $\category{R}$ be a Reedy category and $\category{C}$ be a model category. Then there is a model structure on $[\category{R}, \category{C}]$ in which the 
  \begin{enumerate}
    \item The weak equivalences are object-wise weak equivalences,

    \item The cofibrations are maps $X \to Y$ such that for all $r \in \category{R}$, the map
      \begin{equation*}
        L_{r}Y \amalg_{L_{r}X} X_{r} \to Y_{r}
      \end{equation*}
      is a cofibration in $\category{C}$.

    \item The fibrations are maps $X \to Y$ such that for all $r \in \category{R}$, the map
      \begin{equation*}
        X_{r} \to M_{r}X \times_{M_{r}Y} Y_{r}
      \end{equation*}
      is a fibration in $\category{C}$.
  \end{enumerate}
\end{theorem}

\begin{example}
  Take $\category{R} = \Delta\op$ and $\category{C} = \SSet^{\Kan}$. Then the Reedy model structure has the following classes of morphisms.
  \begin{enumerate}
    \item The weak equivalences are level-wise weak equivalences.

    \item The cofibrations are level-wise monomorphisms.

    \item The fibrations are maps $f\colon X \to Y$ such that the map
      \begin{equation*}
        \Fun(\Delta^{n}, Y) \to \Fun(\partial\Delta^{n}, Y) \times_{\Fun(\partial \Delta^{n}, X)} \Fun(\Delta^{n}, X)
      \end{equation*}
      is a Kan fibration.
  \end{enumerate}
\end{example}

\end{document}
