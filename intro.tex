\documentclass[main.tex]{subfiles}

\begin{document}

\chapter{Introduction}
\label{ch:introduction}

To do for next time (Monday, Jan.\ 17):
\begin{enumerate}
  \item Write up the proofs of the unclear sections of Barwick more clearly

  \item Make precise the diagram
    \begin{equation*}
      \begin{tikzcd}
        \Span(\Fun(\Delta^{1}, \category{C}))^{\otimes}
        \arrow[rr]
        \arrow[ddr]
        \arrow[dr]
        && \Span(\category{C})^{\otimes}
        \arrow[ddl]
        \arrow[dr]
        \\
        & \Span(\Fun(\Delta^{1}, \category{C})^{\otimes})
        \arrow[ddr, crossing over]
        \arrow[rr, crossing over]
        && \Span(\category{C}^{\otimes})
        \arrow[ddl]
        \\
        & \Finp
        \arrow[dr]
        \\
        && \Span(\Finp)
      \end{tikzcd}
    \end{equation*}
    where the bottom squares are pullback.

  \item Think about phrasing this in terms of Segal spaces; who know, maybe it becomes trivial!
\end{enumerate}

\section{Description}
\label{sec:description}

\subsection{Categories of correspondences}
\label{ssc:categories_of_correspondences}

\begin{itemize}
  \item $\category{C} \rightsquigarrow \Corr(\category{C}) = \Span(\category{C})$
    \begin{equation*}
      \begin{tikzcd}[column sep=small]
        && y \times_{x'} y'
        \arrow[dr]
        \arrow[dl]
        \\
        & y
        \arrow[dr]
        \arrow[dl]
        && y'
        \arrow[dr]
        \arrow[dl]
        \\
        x
        && x'
        && x''
      \end{tikzcd}
    \end{equation*}

  \item $\category{C}^{\otimes} \rightsquigarrow \Corr(\category{C})^{\otimes}$ (monoidal structure)

  \item $\category{C} \rightsquigarrow \Corr_{2}(\category{C})$, $\Corr_{n}(\category{C})^{\otimes}$ (where perhaps $\otimes$ is a symmetric monoidal $n$-category)

    \begin{equation*}
      \begin{tikzcd}
        & y
        \arrow[dl]
        \arrow[dr]
        \\
        x
        & z
        \arrow[u]
        \arrow[r]
        \arrow[d]
        \arrow[l]
        & x'
        \\
        & y'
        \arrow[ul]
        \arrow[ur]
      \end{tikzcd}
    \end{equation*}

    TFT: $M \mapsto X^{M}$
    \begin{equation*}
      \begin{tikzcd}
        & M
        \\
        \partial^{+}M
        \arrow[ur, hook]
        && \partial^{-}M
        \arrow[ul, hook]
      \end{tikzcd}
      \qquad\mapsto\qquad
      \begin{tikzcd}
        & X^{M}
        \\
        X^{\partial^{+}M}
        \arrow[ur, hook]
        && X^{\partial^{-}M}
        \arrow[ul, hook]
      \end{tikzcd}
    \end{equation*}
\end{itemize}

Haugseng: Iterated Spans and Classical TFTs
\begin{equation*}
  \Bord_{n}^{\otimes} \rightarrow \Corr_{n}(\category{C})^{\otimes} \overset{?}{\dashrightarrow} \Vect^{\otimes}
\end{equation*}

\subsection{Transfer theories}
\label{ssc:transfer_theories}

Let $F\colon \category{C} \to \category{D}$, and let $x \in \category{C}$.
\begin{equation*}
  \begin{tikzcd}
    & y
    \arrow[dl, swap, "p"]
    \arrow[dr, "q"]
    \\
    x
    && x'
  \end{tikzcd}
\end{equation*}

\section{Motivation for spans}
\label{sec:motivation_for_spans}

Let $X$ and $Y$ be sets, say representing the in and out states of some physical system. For example, imagine a photon sent out by some emitter, passing through a screen with holes, and then being detected. The set $X$ might be
\begin{equation*}
  \{\text{left hole}, \text{middle hole}, \text{right hole}\},
\end{equation*}
and the set $Y$ might be
\begin{equation*}
  \{\text{detected at $A$}, \text{detected at $B$}\}.
\end{equation*}

Consider a set $T$, equipped with maps $s\colon T \to X$ and $t\colon T \to Y$. 
\begin{equation*}
  \begin{tikzcd}
    & T
    \arrow[dl, swap, "s"]
    \arrow[dr, "t"]
    \\
    X
    && Y
  \end{tikzcd}
\end{equation*}
We can think of $T$ as consisting of a set of ways of going from an element of $X$ to an element of $Y$; that is, (using the universal property for products) the fiber $T^{x}_{y} := (s, t)^{-1}(x, y)$ can be thought of as the set of ways of going from $x \in X$ to $y \in Y$. In our example, the fiber $T^{x}_{y}$ would correspond to the set of classical trajectories starting at hole $x$ and ending at detector $y$.

Consider concatenating two such processes.
\begin{equation*}
  \begin{tikzcd}
    & T
    \arrow[dl, swap, "s"]
    \arrow[dr, "t"]
    && T'
    \arrow[dl, swap, "s'"]
    \arrow[dr, "t'"]
    \\
    X
    && Y
    && Z
  \end{tikzcd}
\end{equation*}
Taking the pullback in $\Set$ gives us a set of ways of going from $X$ to $Z$.
\begin{equation*}
  \begin{tikzcd}
    && S
    \arrow[dl, swap, "\sigma"]
    \arrow[dr, "\tau"]
    \\
    & T
    \arrow[dl, swap, "s"]
    \arrow[dr, "t"]
    && T'
    \arrow[dl, swap, "s'"]
    \arrow[dr, "t'"]
    \\
    X
    && Y
    && Z
  \end{tikzcd}
\end{equation*}
That is, the fibers $S^{x}_{z}$ consist of pairs $(t, t')$, where $t'$ is a process from $x$ to $y$ and $t'$ is a process from $y$ to $z$.

For a set $X$, denote by $L(X)$ the free $\C$-vector space on $X$. Consider a span $T\colon X \to Y$ as before. We can assign to $T$ a linear map
\begin{equation*}
  L(T)\colon L(X) \to L(Y)
\end{equation*}

\end{document}
