\documentclass[main.tex]{subfiles}

\begin{document}

\appendix

\section{Appendix}

%\subsection{Bisimplicial sets}
%\label{sss:bisimplicial_sets}
%
%The category $\SSet$ carries two model structures of which we will make frequent use:
%\begin{itemize}
%  \item The \emph{Kan} model structure, which has the following description.
%    \begin{itemize}
%      \item The fibrations are the Kan fibrations.
%
%      \item The cofibrations are the monomorphisms.
%
%      \item The weak equivalences are weak homotopy equivalences.
%    \end{itemize}
%
%  \item The \emph{Joyal} model structure. We will not give a complete description, referring the reader to \cite[Sec.\ 2.2.5]{highertopostheory}. We will make use of the following properties.
%    \begin{itemize}
%      \item The cofibrations are monomorphisms.
%
%      \item The fibrant objects are quasicategories, and the fibrations between fibrant objects are isofibrations, i.e.\ inner fibrations with lifts of equivalences (cf \cite[Cor.\ 2.6.5]{highertopostheory}).
%    \end{itemize}
%\end{itemize}
%
%The category $\D\op$ has a Reedy structure, which gives us, together with the Kan model structure, a model structure on the category $\Fun(\D\op, \SSet)$ with the following properties.
%\begin{itemize}
%  \item The cofibrations are monomorphisms.
%
%  \item The weak equivalences are level-wise weak homotopy equivalences.
%
%  \item The fibrations are \emph{Reedy fibrations} (\hyperref[def:reedy_fibration]{Definition~\ref*{def:reedy_fibration}}).
%\end{itemize}
%
\subsection{Divisibility of bifunctors}
\label{sss:divisibility_of_bifunctors}

In this section, we recall some key results from \cite{qcats_vs_segal_spaces}. We refer readers there for more information.

Let $\odot\colon \mathcal{E}_{1} \times \category{E}_{2} \to \category{E}_{3}$ be a functor. We will say that $\odot$ is \emph{divisible on the left} if for each $A \in \category{E}_{1}$, the functor $A \odot -$ admits a right adjoint $A \backslash -$. In this case, this construction turns out also to be functorial in $A$; that is, we get a functor
\begin{equation*}
  - \backslash -\colon \category{E}_{1}\op \times \category{E}_{3} \to \category{E}_{2}.
\end{equation*}

Analogously, $\odot\colon \mathcal{E}_{1} \times \mathcal{E}_{2} \to \mathcal{E}_{3}$ is \emph{divisible on the right} if for for each $B \in \category{E}_{2}$, the functor $- \odot B$ admits a right adjoint $- / B$. In this case we get a of two variables
\begin{equation*}
  - / -\colon \category{E}_{3} \times \category{E}_{2}\op \to \category{E}_{1}.
\end{equation*}

\begin{example}
  It will be helpful to keep in mind the cartesian product 
  \begin{equation*}
    - \times -\colon \SSet \times \SSet \to \SSet. 
  \end{equation*}
  In this case, both $A \backslash X$ and $X / A$ are the mapping space $X^{A}$.
\end{example}

If $\odot$ is divisible on both sides, then there is a bijection between maps of the following types:
\begin{equation*}
  A \odot B \to X,\qquad A \to X / B,\qquad B \to A \backslash X.
\end{equation*}
In particular, this implies that the functors $X / -$ and $- \backslash X$ are mutually right adjoint.

If both $\category{E}_{1}$ and $\category{E}_{2}$ are finitely complete and $\category{E}_{3}$ is finitely cocomplete, then from a map $u\colon A \to A'$ in $\category{E}_{1}$, a map $v\colon B \to B'$ in $\category{E}_{2}$, and a map $f\colon X \to Y$ in $\category{E}_{3}$, we can build the following maps.
\begin{itemize}
  \item From the square
    \begin{equation*}
      \begin{tikzcd}
        A \odot B
        \arrow[r]
        \arrow[d]
        & A' \odot B
        \arrow[d]
        \\
        A \odot B'
        \arrow[r]
        & A' \odot B'
      \end{tikzcd}
    \end{equation*}
    we get a map
    \begin{equation*}
      u \odot' v\colon A \odot B' \amalg_{A \odot B} A' \odot B \to A' \odot B'.
    \end{equation*}

  \item From the square
    \begin{equation*}
      \begin{tikzcd}
        A' \backslash X
        \arrow[r]
        \arrow[d]
        & A \backslash X
        \arrow[d]
        \\
        A' \backslash Y
        \arrow[r]
        & A \backslash Y
      \end{tikzcd}
    \end{equation*}
    we get a map
    \begin{equation*}
      \langle u \backslash f \rangle\colon A' \backslash X \to A \backslash X \times_{A \backslash Y} A' \backslash Y
    \end{equation*}

  \item From the square
    \begin{equation*}
      \begin{tikzcd}
        X / B'
        \arrow[r]
        \arrow[d]
        & X / B
        \arrow[d]
        \\
        Y / B'
        \arrow[r]
        & Y / B
      \end{tikzcd}
    \end{equation*}
    we get a map
    \begin{equation*}
      \langle f / v \rangle\colon X / B' \to X / B \times_{Y / B} Y / B'.
    \end{equation*}
\end{itemize}

\begin{proposition}
  \label{prop:equivalent_lifting_problems}
  With the above notation, the following are equivalent adjoint lifting problems:
  \begin{equation*}
    \begin{tikzcd}
      A \odot B' \amalg_{A \odot B} A' \odot B
      \arrow[r]
      \arrow[d]
      & X
      \arrow[d]
      \\
      A' \odot B'
      \arrow[r]
      \arrow[ur, dashed]
      & Y
    \end{tikzcd}
    \qquad
    \begin{tikzcd}
      A
      \arrow[r]
      \arrow[d]
      & X / B'
      \arrow[d]
      \\
      A'
      \arrow[r]
      \arrow[ur, dashed]
      & X / B \times_{Y / B} Y / B'
    \end{tikzcd}
  \end{equation*}
  \begin{equation*}
    \begin{tikzcd}
      B
      \arrow[r]
      \arrow[d]
      & A' \backslash X
      \arrow[d]
      \\
      B'
      \arrow[r]
      \arrow[ur, dashed]
      & A \backslash X \times_{A \backslash Y} A' \backslash Y
    \end{tikzcd}
  \end{equation*}
\end{proposition}

\end{document}
